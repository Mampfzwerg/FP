\section{Diskussion}
\label{sec:Diskussion}

Relative Fehler der Mesergebnisse werden im Folgenden gemäß der Formel

\begin{equation}
    \Delta x = \left| \frac{x_\text{mess} - x_\text{theo}}{x_\text{theo}}\right|
\end{equation}

bestimmt. 

Von dem Theoriewert der effektiven Masse $m^*{theo} = \num{0.067}m_e$ \cite{Theo} (S.3465)
weichen die experimentell bestimmten effektiven Massen um

\begin{align*}
    \Delta m^*_1 &= \SI{52.47}{\percent} \\
    \Delta m^*_2 &= \SI{13.27}{\percent}
\end{align*}

ab.
Da die Abweichungen für die beiden dotierten Proben recht unterschiedlich ist, kann neben möglichen Justagefehlern oder
Messfehlern bei der hochreinen Probe, die beide n-dotierten Proben betreffen würden, ein Fehler bei der Messung der ersten 
Probe vorgefallen sein. Beispielsweise könnte die erste Probe leicht beschädigt oder bei deren Messung Unreinheiten der Optiken
vorhanden gewesen sein. Insgesamt allerdings liegen beide Ergebnisse recht nah am Theoriewert.