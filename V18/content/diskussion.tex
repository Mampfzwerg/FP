

Die gemessene Halbwertsbreite $\Delta E_{\sfrac{1}{2}} \approx \SI{0.52 +- 0.05}{\kilo\eV}$ weicht um etwa
\SI{49.47}{\percent} von der theoretischen $\Delta E_{\sfrac{1}{2}, \text{theo}} = \SI{1029.16 +- 0.04}{\eV}$ ab.

Dieser Fehler resultiert unter anderem aus Ablesefehlern und der Näherung des Photopeaks als Gaußverteilung.
Die Näherung weicht für die niedrigen Wertepaare, von denen die Zehntelwertsbreite abhängt, besonders stark ab.
Am restlichen Peak ist ihre Abweichung geringer. \\


\section{Literaturverzeichnis}
\label{sec:Diskussion}

[1] Praktikumsanleitung. Versuch V18 - Der Reinst-Germanium-Detektor als Instrument
der Gamma-Spektroskopie:

\url{http://129.217.224.2/HOMEPAGE/PHYSIKER/BACHELOR/FP/SKRIPT/V18.pdf}.

[2] Laboratoire National Henri Becquerel:

\url{http://www.nucleide.org/Laraweb/index.php}; letzter Zugriff 16.06.2020.
