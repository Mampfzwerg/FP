\section{Diskussion}
\label{sec:Diskussion}

Die Ergebnisse der Energiekalibrierung und Effizienzbestimmung des Detektors anhand des 
Spektrums von $\ce{^{152}}$Europium sind in den Abbildungen \ref{fig:plot5} und \ref{fig:plot6}
veranschaulicht. 
Außerdem sind die statistischen Fehler der Regressionsparameter der Energiekallibrierung mit $\SI{0}{\percent}$ 
und $\SI{1.9}{\percent}$ sehr gering.\\

Bei der Untersuchung des monochromatischen Gammaspektrums von $\ce{^{137}}$Caesium
weicht die gemessene Energie des Photopeaks

\begin{equation*}
    E_\text{Photo} = \SI{661.35 +- 0.05}{\kilo\eV}
  \end{equation*}
  
leicht, um etwa $\SI{0.05}{\percent}$, vom Literaturwert $E_\text{Photo, Lit} = \SI{661.657}{\kilo\eV}$ [2] ab. 

Das Verhältnis zwischen Halb- und Zehntelwertsbreiten des mit einer Gaußverteilung genäherten Photopeaks

\begin{equation}
  \kappa = \frac{\mu_{0, \sfrac{1}{10}}}{\mu_{0, \sfrac{1}{2}}} = \num{1.866}\; .
\end{equation}

weicht um nur $\SI{2.35}{\percent}$ von dem für Gaußverteilungen typischen Wert von $\kappa = \num{1.823}$ ab.

Die daraus bestimmte Energie-Halbwertsbreite $\Delta E_{\sfrac{1}{2}} = \SI{2.2 +- 0.1}{\kilo\eV}$ weicht um etwa
$\SI{113.77}{\percent}$ von der theoretischen $\Delta E_{\sfrac{1}{2}, \text{theo}} = \SI{1029.16 +- 0.04}{\eV}$ ab.
Die beiden Werte teilen jedoch dieselbe Größenordnung.
Die Abweichung resultiert unter anderem aus Ablesefehlern und der Näherung des Photopeaks als Gaußverteilung.

Die Energien, bei denen sich die Comptonkante und der Rückstreupeak befinden wurden zu

\begin{align*}
    E_\text{K} &= \SI{473 +- 10}{\kilo\eV} \\
    E_\text{R} &= \SI{195 +- 10}{\kilo\eV} \; .
\end{align*}

bestimmt und weichen mit $\SI{0.85}{\percent}$ und $\SI{5.81}{\percent}$ minimal von den berechneten Werten

\begin{align*}
    E_\text{K, theo} &= \SI{477.05 +- 0.05}{\kilo\eV} \\
    E_\text{R, theo} &= \SI{184.299 +- 0.004}{\kilo\eV}
\end{align*}

ab.

Der Vergleich der Zählraten bzw. Linieninhalte des Photopeaks und Comptonkontinuums

\begin{align*}
    Z_\text{Photo} &= \num{9510} \\
    Z_\text{Compton} &= \num{33795}
\end{align*}

mit den Absorptionswahrscheinlichkeiten aus Tabelle \ref{tab:mess3} zeigt, dass die Ereignisanzahl im Photopeak 
$3 \sfrac{1}{2}$-mal geringer und die Absorptionswahrscheinlichkeit über $30$-mal kleiner ist, als im Comptonkontinuum.
Daraus ist zu schließen, dass der Comptoneffekt Einfluss auf den Photopeak hat. Compton-gestreute Gammaquanten können im 
Detektor noch photoelektrisch wechselwirken.\\

Bei der Aktivitätsbestimmung von  $\ce{^{125}}$Antimon oder  $\ce{^{133}}$Barium konnten
die in Tabelle \ref{tab:mess5} aufgelisteten Aktivitäten bestimmt werden und die Quelle durch den
Vergleich gemessener Peakenergien mit Referenzsprektren als $\ce{^{133}}$Barium identifiziert werden.
Die gemittelte Aktivität beträgt $\bar{A} = \SI{1180 +- 383}{\becquerel}$.\\

Die Nuklididentifikation eines unbekannten kristallinen Strahlers ergab die in Tabelle \ref{tab:mess7}
aufgelisteten Peakenergien und ließen eine Identifikation der Nuklide 
$\ce{^{234}}$Thorium, $\ce{^{226}}$Radium, $\ce{^{214}}$Plumbum (Blei), $\ce{^{214}}$Bismut und
$\ce{^{210}}$Thallium zu. Diese sind Bestandteile der Zerfallsreihe von $\ce{^{238}Uran}$ und lassen
darauf schließen, dass es sich bei dem unbekannten Strahler um ein Urangestein handelt.


\section{Literaturverzeichnis}
\label{sec:Literaturverzeichnis}

[1] Praktikumsanleitung. Versuch V18 - Der Reinst-Germanium-Detektor als Instrument
der Gamma-Spektroskopie:

\url{https://github.com/bixel/FP14/blob/master/V18_Ge-Detektor/V18.pdf}; letzter Zugriff 30.07.2020

[2] Laboratoire National Henri Becquerel:

\url{http://www.nucleide.org/Laraweb/index.php}; letzter Zugriff 16.06.2020.
