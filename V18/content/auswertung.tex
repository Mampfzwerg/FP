\section{Auswertung}
\label{sec:Auswertung}

%%%%%%%%%%%%%%%%%%%%%%%%%%%%%%%%%%%%%%%%%%%%%%%%%%%%%%%%%%%%%%%%%%%%%%%%%%%%%%%%%%%%%%%%%%%%%%%%%%%%%%%%%%%%%%%%%%%%%%%%%%%%%%%%%%%

\subsection{Energiekalibrierung anhand des Spektrums von $\ce{^{152}}$Europium}

Zur Energiekalibrierung des Detektors wird das Spektrum eines $\ce{^{152}Eu}$-Strahlers aufgenommen, welches in Abbildung
\ref{fig:plot1} zu sehen ist.



Dazu werden anhand der charakteristischen Peaks des Spektrums mittels linearer Regression Peakenergien $E$ [1] und mittlere 
Kanalnummern $\mu_0$ in Beziehung gesetzt.
Die Peaks werden als Gaußverteilungen der Form 

\begin{equation}
  f(x) = a \cdot \text{exp}\left( - \frac{(x-\mu_0)^2}{b}\right) + c
  \label{eqn:gauss}
\end{equation}

genähert. Dazu wird die Funktion \textit{scipy.optimize.curve\_fit} aus der Python-Bibliothek SkiPy verwendet.

Die den Peakenergien $E$ zugeordneten Näherungsparameter sind in Tabelle \ref{tab:mess1} aufgelistet.

Mit den Wertepaaren der Peakenergien $E$ und der mittleren Kanalnummern $\mu$ wird eine lineare Regression der Form

\begin{equation}
  E(\mu) = m \cdot \mu_0 + d
  \label{eqn:eich}
\end{equation}
  
durchgeführt, die in Abildung \ref{fig:plot5} dargestellt ist.
Die Regressionsparameter betragen

\begin{align*}
  m &= \num{0.403 +- 0.000} \\
  d &= \num{-2.683 +- 0.051} \; .
\end{align*}

Die statistischen Fehler der Regressionskonstanten sind mit $\SI{0.0}{\percent}$ und $\SI{1.9}{\percent}$ sehr gering.

\begin{figure}[H]
  \centering
  \includegraphics[scale=0.7]{content/plot1.pdf}
  \caption{Gemessenes Spektrum eines $\ce{^{152}Eu}$-Strahlers, abgeschnitten ab Kanalnummer $\num{4000}$}
  \label{fig:plot1}
\end{figure}

\begin{table}
  \centering
  \caption{Gefittete Parameter der Gaußnäherungen der charakteristischen Peaks des $\ce{^{152}Eu}$-Spektrums}
  \label{tab:mess1}
  \sisetup{table-format=2.1}
  \begin{tabular}{c c c c c}
  \toprule
  $ E \;/\; \si{\kilo\eV}$ & $\mu_0$ & $a$ & $b$ & $c$ \\
  \midrule
        121,78 &  308,80 $\pm$ 0,01 & 3493,0 $\pm$ 15,0 &  3,5 $\pm$ 0,0 & 100,3 $\pm$ 1,3 \\
        244,70 &  613,80 $\pm$ 0,03 &  535,0 $\pm$  9,0 &  4,2 $\pm$ 0,2 &  45,6 $\pm$ 1,1 \\
        344,30 &  860,70 $\pm$ 0,01 & 1140,0 $\pm$  6,0 &  5,5 $\pm$ 0,1 &  24,4 $\pm$ 0,6 \\
        411,12 & 1026,51 $\pm$ 0,09 &   74,7 $\pm$  3,5 &  5,7 $\pm$ 0,6 &  18,9 $\pm$ 0,8 \\
        443,96 & 1107,93 $\pm$ 0,06 &   94,2 $\pm$  3,0 &  6,4 $\pm$ 0,5 &  16,7 $\pm$ 0,5 \\
        778,90 & 1938,62 $\pm$ 0,05 &  143,3 $\pm$  2,4 & 14,3 $\pm$ 0,5 &  11,7 $\pm$ 0,3 \\
        867,37 & 2158,39 $\pm$ 0,17 &   42,5 $\pm$  2,1 & 17,1 $\pm$ 2,0 &  10,7 $\pm$ 0,4 \\
        964,08 & 2398,18 $\pm$ 0,06 &  120,3 $\pm$  2,2 & 19,3 $\pm$ 0,8 &   6,4 $\pm$ 0,5 \\
       1085,90 & 2700,41 $\pm$ 0,14 &   67,5 $\pm$  2,2 & 30,2 $\pm$ 2,3 &   5,3 $\pm$ 0,5 \\
       1112,10 & 2765,02 $\pm$ 0,11 &   88,5 $\pm$  2,6 & 23,2 $\pm$ 1,7 &   5,6 $\pm$ 0,9 \\
       1408,00 & 3499,69 $\pm$ 0,07 &   90,7 $\pm$  1,4 & 28,9 $\pm$ 1,1 &   0,9 $\pm$ 0,3 \\
  \bottomrule
  \end{tabular}
  \end{table}

\begin{figure}[H]
  \centering
  \includegraphics[scale=0.7]{content/plot5.pdf}
  \caption{Lineare Regression der Energie-Kanalnummer-Wertepaare des $\ce{^{152}Eu}$-Strahlers}
  \label{fig:plot5}
\end{figure}


%%%%%%%%%%%%%%%%%%%%%%%%%%%%%%%%%%%%%%%%%%%%%%%%%%%%%%%%%%%%%%%%%%%%%%%%%%%%%%%%%%%%%%%%%%%%%%%%%%%%%%%%%%%%%%%%%%%%%%%%%%%%%%%%%

\subsection{Bestimmung der Effizienz bzw. Vollenergienachweiswahrscheinlichkeit anhand des Spektrums von $\ce{^{152}}$Europium}

Die Effizienz

\begin{equation}
  Q = \frac{4 \pi Z}{\Omega A W t_m}
  \label{eqn:akt}
\end{equation}

wird aus der Zählrate  $Z$, dem Raumwinkel $\Omega$, der Aktivität $A$ des Strahlers, 
der Emmisionswahrscheinlichkeit $W$ und der Messzeit $t_m = \SI{4111}{\second}$ bestimmt.

Der Raumwinkel ergibt sich zu 

\begin{equation}
  \Omega = 2 \pi \left(1 - \frac{l}{\sqrt{l^2 + r^2}}\right) \approx \num{0.0538} \pi \; ,
\end{equation} 

mit dem Radius $r = \SI{2.25}{\centi\meter}$ und dem Abstand $l = \SI{9.5}{\centi\meter}$ zwischen Probe und Detektor.

Die Aktivität am Messtag wird bestimmt als

\begin{equation}
  A(t) = A_0 \cdot \text{exp} \left(- \frac{\text{ln}(2)}{t_{\sfrac{1}{2}}} \cdot t \right)  \; .
\end{equation}

Mit einer Anfangsaktivität $A_0 = \SI{4130 +- 60}{\becquerel}$ am 01.10.2000 und der Halbwertszeit
$t_{\sfrac{1}{2}} = \SI{4943 +- 5}{\day}$ ergibt sich nach einer Zeit von $t = \SI{7177}{\day}$
eine Aktivität von $A_{25.05.2020} = \SI{1510 +- 22}{\becquerel}$.
Der statistische Fehler von $A_{25.05.2020}$ ergibt sich nach 
Gaußscher Fehlerfortpflanzung aus denen von $A_0$ und $t_{\sfrac{1}{2}}$.

Zuletzt sind noch die Zählraten $Z$ der Peaks zu bestimmen. Dazu werden die gefitteten Gaußnäherungen integriert. 
Der konstante Parameter $b$ wird dabei vernachlässigt, da er das Niveau des umgebenden Spektrums angibt und somit nicht zur 
Zählrate $Z$ des Peaks beiträgt. Aus der Integration folgt dann

\begin{equation}
  Z = \int_{-\infty}^\infty \; a \cdot \text{exp}\left( - \frac{(x-\mu_0)^2}{b}\right) \; \text{d}x = a \sqrt{b\pi} \; .
  \label{eqn:rate}
\end{equation}

Der statistische Fehler der Zählrate wird aufgrund des poissonverteilten Zerfallswahrscheinlichkeit als $\sqrt{Z}$ angenommen.

In der erweiterten Tabelle \ref{tab:mess2} sind die errechneten Effizienzen $Q$, Zählraten $Z$ und 
Emmisionswahrscheinlichkeiten $W$ [1] aufgeführt.

Um eine Potenzfunktion mit Zusammenhang zwischen Effizienz und Energie zu bestimmen,
werden die logarithmierte Effizienzen der Peaks gegen deren logarithmierten Energien aufgetragen und eine
lineare Regression der Form

\begin{equation}
  \text{ln}(Q) = n \cdot \text{ln}(\frac{E}{\SI{1}{\kilo\eV}}) + e
\end{equation}

durchgeführt, die in Abbildung \ref{fig:plot6} dargestellt ist.

Es ergeben sich die Regressionsparameter

\begin{align*}
  n &= \num{-0.934 +- 0.035} \\
  e &= \num{-0.742 +- 0.221} \; 
\end{align*}

und somit der Zusammenhang

\begin{equation}
  \label{eqn:eff}
  Q(E) = \text{e}^e \cdot \left(\frac{E}{\SI{1}{\kilo\eV}}\right)^n 
  \approx \num{0.476} \cdot \left(\frac{E}{\SI{1}{\kilo\eV}}\right)^{\num{-0.934}} \; .
\end{equation}

\begin{table}
  \small
  \centering
  \caption{Effizienzen und zu deren Berechnung notwendige Größen des $\ce{^{152}Eu}$-Strahlers}
  \label{tab:mess2}
  \sisetup{table-format=2.1}
  \begin{tabular}{c c c c c c}
  \toprule
  $E \;/\; \si{\kilo\eV}$ & $W \;\text{in}\; \si{\percent}$ & $a$ & $b$ & $Z$ & $Q$ \\
  \midrule
        121,78 & 28,6 & 3493,0 $\pm$ 15,0 &  3,5 $\pm$ 0,0 & 11582,64 $\pm$ 107,62 & 0.0048506 $\pm$  7,37 $\cdot 10^{-5}$ \\
        244,70 &  7,6 &  535,0 $\pm$  9,0 &  4,2 $\pm$ 0,2 &  1943,36 $\pm$  44,08 & 0.0030626 $\pm$  9,98 $\cdot 10^{-5}$ \\
        344,30 & 26,5 & 1140,0 $\pm$  6,0 &  5,5 $\pm$ 0,1 &  4738,72 $\pm$  68,84 & 0.0021417 $\pm$  3,85 $\cdot 10^{-5}$ \\
        411,12 &  2,2 &   74,7 $\pm$  3,5 &  5,7 $\pm$ 0,6 &   316,11 $\pm$  17,78 & 0.0017209 $\pm$ 12,38 $\cdot 10^{-5}$ \\
        443,96 &  3,1 &   94,2 $\pm$  3,0 &  6,4 $\pm$ 0,5 &   422,39 $\pm$  20,55 & 0.0016320 $\pm$  8,56 $\cdot 10^{-5}$ \\
        778,90 & 12,9 &  143,3 $\pm$  2,4 & 14,3 $\pm$ 0,5 &   960,48 $\pm$  30,99 & 0.0008918 $\pm$  2,52 $\cdot 10^{-5}$ \\
        867,37 &  4,2 &   42,5 $\pm$  2,1 & 17,1 $\pm$ 2,0 &   311,50 $\pm$  17,65 & 0.0008883 $\pm$  6,92 $\cdot 10^{-5}$ \\
        964,08 & 14,6 &  120,3 $\pm$  2,2 & 19,3 $\pm$ 0,8 &   936,74 $\pm$  30,61 & 0.0007685 $\pm$  2,40 $\cdot 10^{-5}$ \\
       1085,90 & 10,2 &   67,5 $\pm$  2,2 & 30,2 $\pm$ 2,3 &   657,48 $\pm$  25,64 & 0.0007720 $\pm$  4,03 $\cdot 10^{-5}$ \\
       1112,10 & 13,6 &   88,5 $\pm$  2,6 & 23,2 $\pm$ 1,7 &   755,55 $\pm$  27,49 & 0.0006654 $\pm$  3,27 $\cdot 10^{-5}$ \\
       1408,00 & 21,0 &   90,7 $\pm$  1,4 & 28,9 $\pm$ 1,1 &   864,23 $\pm$  29,40 & 0.0004929 $\pm$  1,41 $\cdot 10^{-5}$ \\
  \bottomrule
  \end{tabular}
  \end{table}

\begin{figure}[H]
  \centering
  \includegraphics[scale=0.7]{content/plot6.pdf}
  \caption{Lineare Regression der logarithmierten Effizienzen abhängig von den logarithmierten Energien}
  \label{fig:plot6}
\end{figure}

%%%%%%%%%%%%%%%%%%%%%%%%%%%%%%%%%%%%%%%%%%%%%%%%%%%%%%%%%%%%%%%%%%%%%%%%%%%%%%%%%%%%%%%%%%%%%%%%%%%%%%%%%%%%%%%%%%%%%%%%%%%%%%%%%%%%%%%%%%%%%

\subsection{Untersuchung des monochromatischen Gammaspektrums von $\ce{^{137}}$Caesium}

Das gemessene Spektrum von $\ce{^{137}Cs}$ ist in den Abbildungen \ref{fig:plot2} und \ref{fig:plot21} dargestellt.
Um die Lage des Photopeaks bzw. der Vollenergielinie zu bestimmen, wird dieser durch eine Gaußglocke genähert, die in Abbildung
\ref{fig:plot22} dargestellt ist.

Die Näherung erfolgt mittels der Funktion \textit{scipy.optimize.curve\_fit} aus der Python-Bibliothek SkiPy und für
das Maximum der Gaußglocke ergibt sich die Kanalnummer

\begin{equation*}
  \mu_{0,\text{Photo}} = \num{1647.72} \approx \num{1648} \; ,
\end{equation*}

die als Zentrum des Photopeaks angenommen wird.

Mit der Energieeichung \eqref{eqn:eich} lässt sich die Energie des Photopeaks als

\begin{equation*}
  E_\text{Photo} = \SI{661.35 +- 0.05}{\kilo\eV}
\end{equation*}

bestimmen.

Der Literaturwert beträgt $E_\text{Photo, Lit} = \SI{661.657}{\kilo\eV}$ [2]. Die Abweichung vom Literaturwert 
von ca. $\SI{0.3}{\kilo\eV}$ liegt über der statistischen Abweichung der Energieeichung, 
ist mit $\SI{0.046}{\percent}$ aber dennoch sehr klein.

Die maximale Ereignisanzahl des gaußgenäherten Photopeaks aus Abbildung \ref{fig:plot22} beträgt $\num{1639.91}$.
Es lassen sich die Halb- und Zehntelwertsbreiten 

\begin{align*}
  \mu_{0, \sfrac{1}{2}} &\approx \num{5.37} \\
  \mu_{0, \sfrac{1}{10}} &\approx \num{10.02}
\end{align*} 

ablesen.

Mit der Energieeichung \eqref{eqn:eich} lassen sich diese in die Energien

\begin{align*}
  \Delta E_{\sfrac{1}{2}} &\approx \SI{0.52 +- 0.05}{\kilo\eV} \\
  \Delta E_{\sfrac{1}{10}} &\approx \SI{1.36 +- 0.05}{\kilo\eV}
\end{align*}

umrechnen.

Dabei ist das Verhältnis zwischen Halb- Zehntelwertsbreiten

\begin{equation}
  \kappa = \frac{\mu_{0, \sfrac{1}{10}}}{\mu_{0, \sfrac{1}{2}}} = \frac{\Delta E_{\sfrac{1}{10}}}{\Delta E_{\sfrac{1}{2}}} = \num{1.8659}
\end{equation}

und weicht damit um nur $\SI{2.35}{\percent}$ von dem für Gaußverteilungen typischen Wert von $\kappa = \num{1.823}$ ab.

Zum Vergleich mit der gemessenen kann die Halbwertsbreite der Gaußverteilung wie folgt genähert und berechnet werden:

\begin{align}
  \Delta E_{\sfrac{1}{2}, \text{theo}} &= \sqrt{8 \cdot \text{ln}(2)} \cdot \sigma \approx \num{2.35} \cdot \sqrt{F E_\text{Photo} E_\text{Ex}} \\
  &= \num{2.35} \cdot \sqrt{\num{0.1}\cdot \SI{661.35 +- 0.05}{\kilo\eV} \cdot \SI{2.9}{\eV}} = \SI{1029.16 +- 0.04}{\eV} \; .
\end{align}

Dabei beschreibt $\sigma$ die Standartabweichung der Gaußverteilung. $E_\text{Ex}$ gibt die
Exzitonenbildungsenergie für Germanium bei einer Temperatur von $\SI{77}{\kelvin}$ an und der Fano-Faktor $F$
berücksichtigt, dass Fluktuationen in der Ladungsträger- bzw. Exzitonenerzeugung
durch Fluktuation der Photonenanregung ausgeglichen werden.

Die gemessene Halbwertsbreite $\Delta E_{\sfrac{1}{2}} \approx \SI{0.52 +- 0.05}{\kilo\eV}$ weicht um etwa
\SI{49.47}{\percent} von der theoretischen $\Delta E_{\sfrac{1}{2}, \text{theo}} = \SI{1029.16 +- 0.04}{\eV}$ ab.

Dieser Fehler resultiert unter anderem aus Ablesefehlern und der Näherung des Photopeaks als Gaußverteilung.
Die Näherung weicht für die niedrigen Wertepaare, von denen die Zehntelwertsbreite abhängt, stärker ab, als am Rest des Peaks. \\

Die Comptonkante liegt etwa bei der Kanalnummer $\mu_{0,\text{K}} = \num{1180}$ und der Rückstreupeak etwa bei
$\mu_{0, \text{R}} = \num{490}$.

Dies entspricht den Energien

\begin{align*}
  E_\text{K} &= \SI{472.86 +- 0.05}{\kilo\eV} \\
  E_\text{R} &= \SI{194.79 +- 0.05}{\kilo\eV} \; .
\end{align*}

Berechnet werden können die Energien nach den Gleichungen \eqref{eqn:Compton} und \eqref{eqn:Kante} zu

\begin{align*}
  E_\text{K, theo} &= E_\text{Photo} \cdot \frac{2 \epsilon}{1 + 2 \epsilon} = \SI{477.05 +- 0.05}{\kilo\eV} \\
  E_\text{R, theo} &= \frac{E_\text{Photo}}{1 + 2 \epsilon} =\SI{184.299 +- 0.004}{\kilo\eV} \; .
\end{align*}

Die Abweichungen zwischen gemessenen und berechneten Energien sind mit $\SI{0.88}{\percent}$ und $\SI{5.69}{\percent}$ klein. 

Zur Bestimmung der Zählraten werden für die des Comptonkontinuums die Ereignisanzahlen linksseitig der Comptonkante
und für die des Photopeaks die innerhalb des Peaks gezählt: 

\begin{align*}
  Z_\text{Compton} &= \num{53860} \\
  Z_\text{Photo} &= \num{9506} \; .
\end{align*}

Nach Gleichung \eqref{eqn:Strahl} ergibt sich für die Absorptionswahrscheinlichkeit:

\begin{equation}
  W(D) = 1 - \text{e}^{- \mu D} \; .
\end{equation}

Der Extinktionskoeffizient lässt sich für verschiedene Energien aus Abbildung \ref{fig:Ext} ablesen.
Für eine Kristalllänge von $D = \SI{4.5}{\centi\meter}$ ergeben sich dann die in Tabelle \ref{tab:mess3}
aufgelisteten Absorptionswahrscheinlichkeiten.

\begin{table}
  \centering
  \caption{Energien, Extinktionskoeffizienten und Absorpionswahrscheinlichkeiten für den Photo- und Comtoneffekt des $\ce{^{152}Eu}$-Strahlers}
  \label{tab:mess3}
  \sisetup{table-format=2.1}
  \begin{tabular}{c c c }
  \toprule
  $ $ & $\text{Photoeffekt}$ & $\text{Comptoneffekt}$ \\
  \midrule
  $E \;/\; \si{\kilo\eV}$                 & 661,35 & 0,00 bis 472,86  \\
  $\mu \;/\; \frac{1}{\si{\centi\meter}}$ & 0,006  & 0,85 bis 0,45 \\
  $W \;/\; \si{\percent}$                 & 2,66   & 86,80 bis 97,82 \\
  \bottomrule
  \end{tabular}
\end{table}

Der Vergleich der Zählraten bzw. Linieninhalte mit den Absorptionswahrscheinlichkeiten zeigt, dass die Ereignisanzahl im Photopeak 
$5 \sfrac{1}{2}$-mal geringer und die Absorptionswahrscheinlichkeit über $30$-mal kleiner ist, als im Comptonkontinuum.
Daraus ist zu schließen, dass der Comptoneffekt Einfluss auf den Photopeak hat. Compton-gestreute Photonen können im 
Detektor noch photoelektrisch wechselwirken.

\begin{figure}{H}
  \centering
  \includegraphics[scale=0.7]{content/plot21.pdf}
  \caption{Gemessenes Spektrum eines $\ce{^{137}Cs}$-Strahlers, abgeschnitten ab Kanalnummer $\num{2000}$. Zur Erkennung
    des Compton-Kontinuums, des Rückstreupeaks und der Compton-Kante wurde ein kleinerer Bereich der Ereignisanzahl gewählt.}
  \label{fig:plot21}
\end{figure}

\begin{figure}{H}
  \centering
  \includegraphics[scale=0.7]{content/plot22.pdf}
  \caption{Gemessener und gaußgenäherter Photopeak eines $\ce{^{137}Cs}$-Strahlers}
  \label{fig:plot22}
\end{figure}

\begin{figure}[H]
  \centering
  \includegraphics[scale=0.7]{content/plot2.pdf}
  \caption{Gemessenes Spektrum eines $\ce{^{137}Cs}$-Strahlers, abgeschnitten ab Kanalnummer $\num{2000}$. Hier ist der
  Photopeak bzw. Vollenergielinie gut zu erkennen.}
  \label{fig:plot2}
\end{figure}



%%%%%%%%%%%%%%%%%%%%%%%%%%%%%%%%%%%%%%%%%%%%%%%%%%%%%%%%%%%%%%%%%%%%%%%%%%%%%%%%%%%%%%%%%%%%%%%%%%%%%%%%%%%%%%%%%%%%%%%%%%%%%%%%%%%

\subsection{Aktivitätsbestimmung von  $\ce{^{125}}$Stibnit oder  $\ce{^{133}}$Barium }

Zur Identifikation, welches Nuklid das in Abbildung \ref{fig:plot3} abgebildete Spektrum besitzt, werden die Peaks gemäß
Gleichung \eqref{eqn:gauss} mithilfe der Funktion \textit{scipy.optimize.curve\_fit} aus der Python-Bibliothek SkiPy
gaußgenähert.

\begin{figure}
  \centering
  \includegraphics[scale=0.7]{content/plot3.pdf}
  \caption{Gemessenes und genähertes Spektrum eines $\ce{^{125}Sb}$- oder  $\ce{^{133}Ba}$-Strahlers,
           abgeschnitten ab Kanalnummer $\num{1200}$}
  \label{fig:plot3}
\end{figure}

Aus den Fitparametren lassen sich nach Gleichung \eqref{eqn:eich} die Energien $E$, nach Gleichung \eqref{eqn:eff} 
die Effizienzen $Q$ und nach Gleichung \eqref{eqn:rate} die Zählraten $Z$ berechnen.

Nach Umstellung der Gleichung \eqref{eqn:akt} ergibt sich die Aktivität

\begin{equation}
  A = \frac{4 \pi Z}{\Omega Q W t_m}
\end{equation}

mit dem Raumwinkel $\Omega \approx \num{0.0538} \pi$ und einer Messzeit von $t_m = \SI{3771}{\second}$.
Diese gefitteten Parameter sind in Tabelle \ref{tab:mess4} und die errechneten Größen in Tabelle \ref{tab:mess5} aufgelistet.

\begin{table}
  \centering
  \caption{Fitparameter der Gaußnäherungen des $\ce{^{125}Sb}$- oder $\ce{^{133}Ba}$-Strahlers}
  \label{tab:mess4}
  \sisetup{table-format=2.1}
  \begin{tabular}{c c c c c}
  \toprule
  $E \;/\; \si{\kilo\eV}$ & $\mu_0$ & $a$ & $b$ & $c$ \\
  \midrule
     81,01 $\pm$ 0,05 & 207,667 $\pm$ 0,023 & 2400 $\pm$ 40 & 3,06 $\pm$ 0,11 & 99,0 $\pm$ 7,0 \\
    276,39 $\pm$ 0,05 & 692,477 $\pm$ 0,026 &  301 $\pm$  5 & 4,08 $\pm$ 0,15 & 20,0 $\pm$ 0,9 \\
    302,84 $\pm$ 0,05 & 758,120 $\pm$ 0,016 &  685 $\pm$  6 & 4,53 $\pm$ 0,10 & 17,4 $\pm$ 1,3 \\
    355,98 $\pm$ 0,05 & 889,976 $\pm$ 0,008 & 1769 $\pm$  7 & 5,46 $\pm$ 0,05 &  8,9 $\pm$ 1,6 \\
    383,82 $\pm$ 0,05 & 959,058 $\pm$ 0,035 &  231 $\pm$  4 & 5,37 $\pm$ 0,23 &  6,3 $\pm$ 0,9 \\
  \bottomrule
  \end{tabular}
\end{table}

\begin{table}
  \centering
  \caption{Aktiviäten und zu deren Berechnung notwendige Größen des $\ce{^{125}Sb}$- oder $\ce{^{133}Ba}$-Strahlers}
  \label{tab:mess5}
  \sisetup{table-format=2.1}
  \begin{tabular}{c c c c c}
  \toprule
  $E \;/\; \si{\kilo\eV}$ & $W \;\text{in}\; \si{\percent}$ & $Z$ & $Q$ & $A \;/\; \si{\becquerel}$ \\
  \midrule
     81,01 $\pm$ 0,05 & 34,1 & 7440 $\pm$ 180 & 0,0079 $\pm$ 0,0021 &   550 $\pm$  150 \\
    276,39 $\pm$ 0,05 &  0,5 & 1078 $\pm$  27 & 0,0025 $\pm$ 0,0007 & 17000 $\pm$ 5000 \\
    302,84 $\pm$ 0,05 & 18,3 & 2580 $\pm$  40 & 0,0023 $\pm$ 0,0007 &  1200 $\pm$  400 \\
    355,98 $\pm$ 0,05 & 62,1 & 7330 $\pm$  40 & 0,0020 $\pm$ 0,0006 &  1200 $\pm$  400 \\
    383,82 $\pm$ 0,05 &  8,9 &  949 $\pm$  26 & 0,0018 $\pm$ 0,0006 &  1140 $\pm$  350 \\
  \bottomrule
  \end{tabular}
\end{table}

Aus dem Vergleich der berechneten Energien mit den Literaturwerten der Energiespektren von $\ce{^{125}Sb}$ und $\ce{^{133}Ba}$
lässt sich erkennen, dass die gemessenen Energien denen des $3$., $5$., $6$., $7$., und $8$. Peaks des $\ce{^{133}Ba}$-Spektrums
entsprechen. Die Energien und Emmisionswahrscheinlichkeiten des Referenzspektrums sind in Tabelle \ref{tab:mess6} aufgeführt. \\

\begin{table}
  \centering
  \caption{Literaturwerte der Peakenergien und Emmisionswahrscheinlichkeiten des $\ce{^{133}Ba}$-Strahlers [1]}
  \label{tab:mess6}
  \sisetup{table-format=2.1}
  \begin{tabular}{c c}
  \toprule
  $E \;/\; \si{\kilo\eV}$ & $W \;\text{in}\; \si{\percent}$ \\
  \midrule
     53,16 &  2,2 \\
     79,62 &  2,6 \\
     81,00 & 34,1 \\
    160,61 &  0,6 \\
    223,25 &  0,5 \\
    302,85 & 18,3 \\
    356,02 & 62,1 \\
    383,85 &  8,9 \\
  \bottomrule
  \end{tabular}
\end{table}

Zur Bestimmung der Aktivität $A$ des vorliegenden $\ce{^{133}Ba}$-Strahlers wird der Mittelwert der Aktivitäten der letzten drei
Peaks genommen, da die ersten beiden Werte stark nach unten und oben abweichen.
Die mittlere Aktivität beträgt $\bar{A} = \SI{1180 +- 383}{\becquerel}$.

%%%%%%%%%%%%%%%%%%%%%%%%%%%%%%%%%%%%%%%%%%%%%%%%%%%%%%%%%%%%%%%%%%%%%%%%%%%%%%%%%%%%%%%%%%%%%%%%%%%%%%%%%%%%%%%%%%%%%%%%%%%%%%%%%%%%%%

\subsection{Nuklididentifikation %und Aktivitätsbestimmung 
            eines unbekannten kristallinen Strahlers}

Das vollständige Spektrum des Kristalls ist in Abbildung \ref{fig:plot41} abgebildet.

\begin{figure}
  \centering
  \includegraphics[scale=0.7]{content/plot41.pdf}
  \caption{Gemessenes Spektrum eines unbekannten kristallinien Strahlers}
  \label{fig:plot41}
\end{figure}

Zur Untersuchung der sieben signifkanten Peaks werden diese gemäß Gleichung \eqref{eqn:gauss} 
mithilfe der Funktion \textit{scipy.optimize.curve\_fit} aus der Python-Bibliothek SkiPy gaußgenähert.
Dies ist in Abbildung \ref{fig:plot4} zu sehen.

\begin{figure}
  \centering
  \includegraphics[scale=0.7]{content/plot4.pdf}
  \caption{Gemessenes und gaußgenähertes Spektrum eines unbekannten kristallinen Strahlers,
           abgeschnitten ab Kanalnummer $\num{1700}$}
  \label{fig:plot4} 
\end{figure}

Die Fitparameter, Energien $E$, sowie die identifizierten Nuklide, deren Referenzenergien $E_\text{N}$ und 
Emmisionswahrscheinlichkeiten $W$ [1] sind in Tabelle \ref{tab:mess7} aufgelistet. \\

\begin{table}
  \centering
  \caption{Fitparameter, Energien, Emmisionswahrscheinlichkeiten und identifizierte Nuklide des unbekannten Strahlers}
  \label{tab:mess7}
  \sisetup{table-format=2.1}
  \begin{tabular}{c c c c c c c}
  \toprule
  $\mu_0$ & $a$ & $b$ & $c$ & $E \;/\; \si{\kilo\eV}$ & $E_\text{N} \;/\; \si{\kilo\eV}$ & Nuklid \\
  \midrule
   197,82 &  -2054373,96 &   -8490,82 &  2056192,47 &  77,04 $\pm$ 0,05 &   - &               - \\
   237,03 &  -4951747,22 &  -65164,91 &  4953165,50 &  92,84 $\pm$ 0,05 &  93 & $\ce{^{234}Th}$ \\
   468,05 & -12252338,86 &  -90236,30 & 12254032,57 & 185,94 $\pm$ 0,05 & 186 & $\ce{^{226}Ra}$ \\
   607,09 & -22934205,13 & -180267,05 & 22935710,52 & 241,97 $\pm$ 0,05 & 242 & $\ce{^{214}Pb}$ \\
   739,04 & -76310651,18 & -294905,26 & 76313418,70 & 295,15 $\pm$ 0,05 & 295 & $\ce{^{214}Pb}$ \\  
   879,60 & -27221198,74 & -143724,37 & 27224703,53 & 351,80 $\pm$ 0,05 & 352 & $\ce{^{214}Pb}$ \\
  1517,94 &  -8988277,11 & -103001,94 &  8990390,62 & 609,05 $\pm$ 0,05 & 609 & $\ce{^{214}Bi}$ \\
  \bottomrule
  \end{tabular}
\end{table}

Mit den gefundenen Energien lassen sich die Nuklide $\ce{^{234}Th}$, $\ce{^{226}Ra}$, $\ce{^{214}Pb}$ und
$\ce{^{214}Bi}$ identifizieren. Diese sind Bestandteile der Zerfallsreihe von $\ce{^{238}Uran}$.
Die niedrigste Peakenergie kann zwar keinem der Nuklide der Zerfallsreihe zugeordnet werden, dennoch kann davon ausgegangen
werden, dass es sich bei der unbekannten Strahlungsquelle um ein Urangestein handelt.
Der nicht dazu passende Peak kann entweder aus einer anderen Zerfallsreihe stammen oder es handelt sich um eine Störung
wie beispielsweise einen Rückstreupeak einer anderen Linie.


%Aus den Fitparametren lassen sich nach Gleichung \eqref{eqn:eff} die Effizienzen $Q$, nach Gleichung \eqref{eqn:rate} 
%die Zählraten $Z$ berechnen und nach Gleichung \eqref{eqn:akt} die Aktivitäten $A$ berechnen.
%Die Messzeit beträgt $t_m = \SI{4797}{\second}$ und der Raumwinkel für einen Abstand zwischen Quelle und Detektor von
%$l = \SI{1.5}{\centi\meter}$: $\Omega \approx \num{0.8906} \pi$.
%
%mit dem Raumwinkel $\Omega \approx \num{0.0538} \pi$ und einer Messzeit von $t_m = \SI{3771}{\second}$.


