\section{Aufbau und Durchführung}
\label{sec:Durchführung}

\subsection{Aufbau eines Germanium Detektors}

\begin{figure}
    \centering
    \includegraphics[scale=0.3]{content/aufbau.png}
    \caption{Blockschaltbild des verwendeten Spektrometers [1].}
    \label{fig:aufbau}
\end{figure}


In Abbildung \ref{fig:aufbau} ist das Blockschaltbild und damit der prinzipielle Aufbau des hier verwendeten Germanium Detektors zu sehen. Ein solche Detektor ermöglicht es 
Spannungsimpulse mit einer Höhe proportional zur Photonenenergie zu erzeugen und abzuspreichern. Dazu wird der in der ladungsträgerverarmten Zone erzeugte 
Ladungsimpuls durch elektirsche Integration mittels eines kapazitiv gekoppelten Operationsverstärkers in ein Spannungspegel umgewandelt. Der Integrationskondensator
wird regelmäßig nach jedem Quantennachweis per optoelektrischer Rückkopplung entladen, das ansonsten das Ausgangspotential des Operationsverstärkers stufenförmig 
ansteigen würde. \\
Die Bandbreite des Verstärkers muss gut angepasst sein, damit keine Rauschspannung entsteht und alle wesentlichen Komponenten des Signals erfasst werden. Dies 
geschieht mittels Hoch- und Tiefpassfiltern. Der Detektor darf nur an die Hochspannung angeschlossen werden, wenn dieser abgekühlt ist. Darum ist ein Temperaturfühler
im Detektorgehäuse eingebaut, der einen Termperaturwächter steuert, sodass die Detektorspannung nicht an einen warmen Kristall gelegt wird. Die Hochspannung 
darf nur langsam geändert werden, da an der Eingangsstufe des Vorverstärkers sonst hohe Spannungen auftreten, die diesen zerstören können. Die Signalspannung wird nun
in einem Vierkanalanalysator gemäß ihrer Höhe in Kanäle einsortiert, welche schließlich auf dem Rechner als Spektrum dargestellt werden. 

\subsection{Messprogramm}

Zur Kallibrierung des Detektors wird das Spektrum von Europium aufgenommen. Aufgrund seiner zahlreichen und bekannten Peaks können mit Hilfe seiner Aktivität $A$
die Effizienz $Q$, sowie die Energiezuordnung der Kanäle berechnet werden. Anschließend wird das Spektrum von Caesium aufgenommen und danach das Spektrum einer
Antimon und einer Barium Quelle. Schließlich wird das Spektrum eines unbekannten Kristalls aufgenommen. 
