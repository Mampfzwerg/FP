\section{Auswertung}
\label{sec:Auswertung}

\subsection{Daten der verwendeten Helmholtzspulenpaare}

Die verwendeten Spulen haben folgende Abmessungen, wobei der Radius mit $R$ und die Windungszahl mit $N$ gegeben ist:

\begin{align*}
  R_\text{Horizontalfeldspule} &= \SI{15.79}{\centi\meter}\,,\\
  R_\text{Sweepfeldspule} &= \SI{16.39}{\centi\meter}\,,\\
  R_\text{Vertikalfeldspule} &= \SI{11.735}{\centi\meter}\,,\\
  N_\text{Horizontalfeldspule} &= \num{154}\,,\\
  N_\text{Sweepfeldspule} &= \num{11}\,,\\
  N_\text{Vertikalfeldspule} &= \num{20}\,.\\  
\end{align*}

\subsection{Korrektur des vertikalen Erdmagnetfeldes}

Zur Korrektur des vertikalen Erdmagnetfeldes wird der Aufbau entsprechend \label{sec:Durchführung} ausgerichtet und das Feld der Vertikalfeldspule auf $T_\text{vert} = \SI{0.202}{\ampere}$
eingestellt. Gemäß \eqref{eqn:helm} kompensiert das Feld der Vertikalfeldspule also ein vertikales Erdmagnetfeld von 

\begin{equation*}
  B_\text{vert} = \SI{30.96}{\micro\tesla}\,.
\end{equation*}

\subsection{Vermessen des Magnetfeldes in Abhängigkeit von der Resonanzfrequenz}

Die im Versuch gemessenen Ströme in Abhängigkeit von der Frequenz des Wechselfeldes sind in Tabelle \ref{tab:mess1} angegeben.

\begin{table}
  \centering
  \caption{Die für die Horiuontalfeld- und Sweepspulen gemessenen Ströme $I_\text{H}$ und $I_\text{S}$ für die Resonanzen 1 und 2 und daraus errechneten Magnetfeldstärken $B$ in Abhängigkeit der angelegten Frequenz $f$.}
  \label{tab:mess1}
  \sisetup{table-format=2.1}
  \begin{tabular}{c c c c c c c}
  \toprule
  $f \,/\, \si{\kilo\hertz}$ & $I_\text{H1} \,/\, \si{\ampere}$ & $I_\text{S1} \,/\, \si{\ampere}$
  & $B_\text{hor,1} \,/\, \si{\micro\tesla}$ & $I_\text{H2} \,/\, \si{\ampere}$ & $I_\text{S2} \,/\, \si{\ampere}$
  & $B_\text{hor,2} \,/\, \si{\micro\tesla}$\\
  \midrule 
       100& & & & & & \\
       200& & & & & & \\
       300& & & & & & \\
       400& & & & & & \\
       500& & & & & & \\
       600& & & & & & \\
       700& & & & & & \\
       800& & & & & & \\
       900& & & & & & \\
      1000& & & & & & \\
  \bottomrule
  \end{tabular}
\end{table}

Hierbei bezeichnet $I_\text{H}$, $I_\text{S}$ und $B_\text{hor}$ die für die jeweiligen Resonanzen eingestellten Stromstärken an Horizontal- und 
Sweepspule, sowie das daraus resultierende gesamte horizontale Magnetfeld. Dieses ist durch die lineare Superposition der einzelnen Magnetfelder
errechenbar.\\

%\begin{figure}
%  \centering
%  \includegraphics{plot.pdf}
%  \caption{Plot.}
%  \label{fig:plot}
%\end{figure}
