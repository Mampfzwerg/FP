\section{Diskussion}
\label{sec:Diskussion}

Für die Ermittlung der vertikalen Erdmagnetfelds ergibt sich durch die Errechnung aus dem Kompensationsfeld durch die vertikale Spule
mit Hilfe der Helmholtzgleichung ein Wert von $\SI{35.09}{\micro\tesla}$. Aus der linearen Ausgleichsrechnung ergibt 
sich hingegen ein Wert von $\num{25.5\pm 1.2}\si{\micro\tesla}$. \\
Ein Literaturwert für diesen Wert liegt bei etwa $\SI{22}{\micro\tesla}$. Die Abweichung kann durch Messungenauigkeiten erklärt werden. Der 
ermittelte Wert ist demnach gut. \\
Die Landefaktoren werden zu

\begin{align*}
    g_\text{F1} &= \num{0.519\pm 0.008},\\
    g_\text{F2} &= \num{0.336\pm 0.002}.
\end{align*}

bestimmt. Literaturwerte liegen bei 

\begin{align*}
    g_\text{F1} &= \num{0.5},\\
    g_\text{F2} &= \frac{1}{3}.
\end{align*}

Die Resonanz 1 kann damit dem Isotop $^{87}\text{Rb}$ und die Resonanz 2 dem Isotop $^{85}\text{Rb}$ zugeordnet werden. Die 
relativen Abweichungen sind dabei im Rahmen von $\SI{2}{\percent}$.\\
Die Kernspins werden zu 

\begin{align*}
    I_1 &= \num{1.416\pm 0.030}\,,\\
    I_2 &= \num{2.477\pm 0.019}\,,
  \end{align*}

bestimmt. Die theoretischen Kernspins liegen bei

\begin{align*}
    I_1 &= \num{1.5}\,,\\
    I_2 &= \num{2.5}\,.
\end{align*}

Das relative Isotopenverhältnis ergibt sich zu $\num{0.4}$. Das in der Natur vorkommende Verhältnis liegt bei $\num{0.386}$. Demnach ist 
das gemessene Verhältnis sehr nahe an dem natürlichen, obwohl das Rubidium-Gasgemisch angereichert ist. Das Ablesen der Minima am Osilloskop 
kann fehlerbehaftet sein und ist recht ungenau. \\
Die Betrachtung des quadratischen Zeeman-Effektes, der um drei beziehungsweise 26
Größenordnungen geringer ist als der lineare, zeigt, dass die getroffenen Näherungen
hinreichend sind.

\section{Literaturverzeichnis}

[1] Roland Gersch, Fatma Kul, Tanja Striepling, Christian Stromenger. \textit{Optisches Pumpen am Rubidium.}

\url{http://www.roland-gersch.de/labs/fprubidium.pdf} \\

[2] TU Dortmund. Praktikumsanleitung. \textit{V21 - Optisches Pumpen.}

\url{https://moodle.tu-dortmund.de/pluginfile.php/1136039/mod_resource/content/2/V21.pdf}.

