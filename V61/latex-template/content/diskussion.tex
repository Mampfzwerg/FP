\vspace{150pt}

\section{Diskussion}
\label{sec:Diskussion}


\subsection{Stabilitätsbedingung}

Die experimentell gemessenen maximalen Resonatorlängen

\vspace{-15pt}
\begin{align*}
    \text{  plan + konkav:} \qquad L_\text{pk,theo} &= \SI{1.4}{\meter} \: ,\\
    \text{konkav + konkav:} \qquad L_\text{kk,theo} &= \SI{2.8}{\meter}
\end{align*}

weichen um $\SI{40.43}{\percent}$ und $\SI{64.18}{\percent}$ von den theoretischen 

\vspace{-15pt}
\begin{align*}
    \text{  plan + konkav:} \qquad L_\text{pk,exp} &= \SI{0.834}{\meter} \: ,\\
    \text{konkav + konkav:} \qquad L_\text{kk,exp} &= \SI{1.003}{\meter} \: 
\end{align*}

ab. Die Abweichungen rühren aus einem systematischem Fehler.
Die Nachjustierung wurde nicht ausreichend durchgeführt.
Dies führte zu einem Fehlschluss, die Stabilitätsbedingung sei nicht mehr erfüllt.
Eine Kontrolle mit den Theoriewerten zeigt jedoch, dass die Resonatoren jeweils
auch doppel so lang noch stabil wären.

\subsection{Wellenlänge}

Die Wellenlänge wurde experimentell zu

\vspace{-5pt}
\begin{equation*}
    \bar{\lambda} = \SI{658 +- 34}{\nano\meter}
\end{equation*}

bestimmt und weicht um $\SI{3.98}{\percent}$ vom Theoriewert

\vspace{-5pt}
\begin{equation*}
    \bar{\lambda} = \SI{632.8}{\nano\meter}
\end{equation*}

aus Abbildung \ref{fig:pump} ab. Die Abweichung ist trotz Messunsicherheiten der
Abmessungen der Versuchsgeometrie und einer nicht idealen Orthogonalität zwischen
Schirm und optischer Achse sehr gering.

\subsection{Transversale Moden}

Die TEM$_{00}$-Grundmode war wesentlich leichter zu vermessen als die TEM$_{01}$-Mode,
da der Draht durch leichte Bewegungen den Laser häufiger unterbrach.
Auch passen die Messergebnisse der Grundmode gut zur theoretischen Gaußverteilung,
während für die Mode höherer Ordnung ein sehr kleines Photosignal nur ungenau 
gemessen werden konnte. Kleine Änderungen der umgebenden Lichteinstrahlung 
konnten die Messung leicht beeinflussen. Daher weichen die Daten stärker
von der theoretischen Verteilung ab.

\subsection{Polarisation}

Der in Abbildung \ref{fig:pol} zu sehende periodische Verlauf
der gemessenen Intensität entspricht der Erwartung einer Periodizität von $2\pi$.
Die konstante Phasenverschiebung resultiert aus der nicht perfekt parallelen Ausrichtung
des Lasers. Weiterhin ist die ungenaue Messung des Photostroms eine weitere Fehlerquelle.

\subsection{Longitudinale Moden}

Die Abweichungen der gemittelten Modenabstände betragen

\vspace{-15pt}
\begin{align*}
    L = \SI{ 69.2}{\centi\meter}: \quad \bar{\Delta f} &= \SI{215.5}{\mega\hertz} & &\SI{0.51}{\percent}\, , \\  
    L = \SI{ 84.8}{\centi\meter}: \quad \bar{\Delta f} &= \SI{174.5}{\mega\hertz} & &\SI{1.28}{\percent}\, , \\  
    L = \SI{100.3}{\centi\meter}: \quad \bar{\Delta f} &= \SI{150}{\mega\hertz}   & &\SI{0.36}{\percent}\, ,  
\end{align*}

von den theoretisch Werten

\vspace{-15pt}
\begin{align*}
    L &= \SI{ 69.2}{\centi\meter}: & \Delta f_\text{theo} &= \SI{216.61}{\mega\hertz}\, , \\  
    L &= \SI{ 84.8}{\centi\meter}: & \Delta f_\text{theo} &= \SI{176.76}{\mega\hertz}\, , \\  
    L &= \SI{100.3}{\centi\meter}: & \Delta f_\text{theo} &= \SI{149.46}{\mega\hertz}\, .  
\end{align*}

und sind nur sehr geringfügig.
Die Doppler-Verbreiterung der dominanten Frequenz des Neon-Übergang konnte zu

\begin{equation}
    \partial f_D = \frac{2f_0}{c} \sqrt{\frac{2RT\: \text{ln}(2)}{M}} = \SI{1314}{\mega\hertz}
\end{equation}

berechnet werden und enthält damit für die untersuchten Resonatorlängen 
die Frequenzen von $6$, $7,5$ oder fast $9$ Moden.
