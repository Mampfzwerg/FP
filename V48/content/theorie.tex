\section{Zielsetzung}

Ziel des Versuches ist es, die Dipolrelaxation bei Ionenkristallen zu untersuchen. Explizit sind die Aktivierungsenergie $W$ und die charakteristische 
Relaxationszeit $\tau_0$ zu bestimmen.

\section{Theorie}
\label{sec:Theorie}

Zunächst werden die Kaliumbromid $\symup{K}^{+}\symup{B}\symup{r}^{-}$ sowie Cäsiumiodid $\symup{Cs}^{+}\symup{J}\symup{r}^{-}$ Proben betrachtet, welche 
jeweils mit Strontium dotiert werden. 

\subsection{Kristallstruktur und Dipole}

Beide proben kristallisieren in einem kubischen Gitter. Dabei nimmt das Kaliumbromid eine NaCl-Struktur an. Diese ist oktaedrisch und es sind je 6 Nachbarionen um jedes 
negativ geladene Brom angeordnet. Die Gitterkonstante beträgt $a = \SI{6.598}{\angstrom}$. Das Cäsiumiodid hingegen kristallisiert in einer CsCl-Struktur, bei der 
ein negatives Jodion von jeweils 8 Cäsiumionen umgeben ist. Die Gitterkonstante von Cäsiumiodid beträgt $a = \SI{4.566}{\angstrom}$.\\
Ein Ionenkristall ist aus abwechselnd positiv und negativen Ionen aufgebaut und somit von außen betrachtet neutral. Die Dotierung fügt doppelt positiv geladene 
Strontiumionen hinzu, sodass Kationen-Leerstellen entstehen, um die lokale Ladungsneutralität zu erhalten.\\
Die Richtung des Dipols wird durch den Vektor zwischen Strontium und Leerstelle bestimmt. Da die Gitterplätze diskret sind, sind auch die Ausrichtungsachsen diskret. 
Unter einer Temperatur $T$ von ungefähr $\SI{500}{\degree}$ ist eine Richtungsänderung durch Leerstellendiffusion möglich, welche bei der Aktivierungsenergie $W$
auftritt. Diese Energie ist eine materialspezifische Größe. Die temperaturabhängige Energieverteilung, die diese Arbeit leisten kann, ist durch die Stefan-Boltzmann-Verteilung
gegeben, welche die Form 

\begin{equation*}
    E \approx \exp{\left(\frac{W}{k_B T}\right)}
\end{equation*}

hat. Die Relaxationszeit ist dann dementsprechend mit 

\begin{equation*}
    \tau\left(T\right) = \tau_0 \exp{\left(\frac{W}{k_B T}\right)}
\end{equation*}

gegeben. Dabei entspricht die charakteristische Relaxationszeit $\tau_0$ der Relaxationszeit für unendlich große Temperaturen. 

\subsection{Messmethodik}

Die Probe befindet sich innerhalb eines Plattenkondensators. Durch das Anlegen eines starken äußeren Feldes wird die Probe zunächst 
polarisiert. Der Anteil 

\begin{equation}
    x'\left(T\right) = \frac{pE}{3k_BT}
    \label{eqn:DipolAnteil}
\end{equation}

der Dipole ist nun ausgerichtet. Dabei ist $p$ das Dipolmoment. Als nächster Schritt wird die Temperatur auf $T_0$ abgesenkt, sodass die Polarisation der 
Probe bestehen bleibt. Das äußere Feld wird abgestellt und die übrige Ladung durch Kurzschließen entfernt. Der folgende Effekt der Dipolrelaxation
hat nun einen induzierten Depolarisationsstrom $T_\text{relax}\left(t\right)$ zur Folge, der eintritt, sobald das Material wieder geheizt wird. 
Die Heizrate ist linear mit 

\begin{equation*}
    b = 2 \si{\kelvin\per\minute},
\end{equation*}

sodass sich eine Temperaturentwicklung von

\begin{equation*}
    T(t) = T_0 + b\cdot t(min)
\end{equation*}

ergibt. $I(t)$ kann beschrieben werden durch 

\begin{equation*}
    I(t) = x'\left(T_{\symup{p}}\right)p\frac{\symup{d}N}{\symup{d}t},
\end{equation*}

wobei $ = x'\left(T_{\symup{p}}\right)$ den Anteil $x'$ der polarisierten Elemente bei der Polarisationstemperatur $T_{\symup{p}}$ und $\frac{\symup{d}N}{\symup{d}t}$ 
die pro Zeit und Volumeneinheit relaxierenden Dipole darstellt. Unter Verwendung der Gleichung \eqref{eqn:DipolAnteil} ergibt sich

\begin{equation*}
    x'\left(T_{\symup{p}}\right)p = \frac{p^2 E}{3k_B T_{\symup{p}}}.
\end{equation*}

Da die Dipolrelaxation ein thermisch aktiver Prozess ist, stellt sich für $\frac{\symup{d}N}{\symup{d}t}$ eine Proportionalität zu $N$ mit der Relaxationsfrequenz 
$\frac{1}{\tau}$ als Proportionalitätsfaktor gemäß 

\begin{equation*}
    \frac{\symup{d}N}{\symup{d}t} = -\frac{N}{\tau\left(T\right)}
\end{equation*}

ein. Eine Integration dieses Terms führt zu 

\begin{equation*}
    N = N_{\symup{p}}\exp{\left(-\frac{1}{b}\int_{T_0}^{T}\frac{\symup{d}T'}{\tau\left(T'\right)}\right)}
\end{equation*}

mit $N_{\symup{p}}$ als Zahl der zu Beginn des Ausheizens vorhandenen orientierten Dipole pro Volumeneinheit. Für den Depolarisationsstrom $I(T)$ ergibt sich

\begin{equation*}
    I\left(T\right) = \frac{p^2E}{3k_B T_{\symup{p}}}\frac{N_{\symup{p}}}{\tau_0} \exp{\left(-\frac{1}{b\tau_0}\int_{T_0}^{T}\exp{\left(-\frac{W}{k_B T'}\right)}\symup{d}T'\right)}\exp{\left(-\frac{W}{k_B T}\right)}
\end{equation*}

Es wird angenommen, dass das Maximum dieses Stroms bei $T_{\text{max}}$ lieft. Diese ist unabhängig von der Starttemperatur.

\begin{equation*}
    T_m^2 = \frac{bW\left(T_{\text{max}}\right)}{k}
\end{equation*}

Für den Grenzwert kleiner Temperaturen gilt damit 

\begin{equation*}
    \ln{\left(I\left(T\right)\right)} = \text{const.}-\frac{W}{k_B T }.
\end{equation*}

Das Intergral über den Strom ist proportional zu der Zahl der Dipole pro Volumen. Es gilt somit 

\begin{equation*}
    \int_{T_0}^{T_1} I(T)\symup{d}T = P_0 A = \frac{Np^2E}{3k_B T}
\end{equation*}

mit Querschnittsfläche $A$ der Probe, und $P_0$ als Polarisation bei der Temperatur $T_0$. Aus der Integration über die im Material 
verbliebenen, noch nicht ausgerichteten Dipole und die Multiplikation mit der Ausrichtungsrate ergibt sich 

\begin{equation*}
    \tau{T} = \tau_0 \exp{\left(\frac{W}{k_B T}\right)} = \int_{I\left(T\right)}^{\infty} \frac{I\left(t'\right)\symup{d}t'}{I\left(T\right)},
\end{equation*}

und 

\begin{equation*}
    \ln{\left(\tau\left(T\right)\right)} = \ln{\left(\tau_0\right)} + \frac{W}{k_B T} = \left(\ln{\int_{I\left(T\right)}^{\infty}I\left(T'\right)\symup{d}T'}\right)-\ln{\left(I\left(T\right)\right)}.
\end{equation*}

Dieses Integral ist im Versuch graphisch zu bestimmen. Diese Gleichung ist unabhängig von der Heizrate.

\subsection{Berechnung der Aktivierungsenergie $W$}

Es werden zwei verschieden Ansätze zur Berechnung der Aktivierungsenergie $W$ betrachtet. Das erste Verfahren nimmt für das Integral

\begin{equation*}
    \int_{T_0}^T \exp{\left(-\frac{W}{k_B T}\right)} \approx 0
\end{equation*}

an. Diese Nährung folgt aus der Nährung für Temperaturen unterhalb der Aktivierungsenergie $W$. Für den Strom ergibt sich somit: 

\begin{equation*}
    I\left(T\right) \approx \frac{p^2 E}{3 k_B T_{\symup{p}}}\frac{N_{\symup{p}}}{\tau_0}\exp{\left(-\frac{W}{k_B T}\right)}.
\end{equation*}

Wird nun eine logarithmische Darstellung gewählt, kann somit $W$ berechnet werden. \\
Das zweite Verfahren bestimmt $W$ über den gesamten Kurvenverlauf. Für die zeitliche Änderung der Polarisation gilt

\begin{equation*}
    \frac{\symup{d}P}{\symup{d}t} = -\frac{P\left(t\right)}{\tau\left(T\left(t\right)\right)}.
\end{equation*}

Diese erzeugt einen äußeren Strom pro Probenquerschnitt $F$ 

\begin{equation*}
    \frac{\symup{d}P}{\symup{d}t} = \frac{I\left(t\right)}{F}.
\end{equation*}

Da $T$ eine lineare Funktion von $t$ sein soll, ergibt sich aus den vorangegangenen Gleichungen 

\begin{equation*}
    \tau\left(T\right) = \frac{\int_T^\infty j\left(T'\right)\symup{d}T'}{bI\left(T\right)}
\end{equation*}

und mit Ersetzen von $\tau\left(T\right)$

\begin{equation*}
    \frac{W}{k_B T} = \ln{\left(\frac{\int_T^\infty I\left(T'\right)\symup{d}T'}{bI\left(T\right)\tau_0}\right)}.
\end{equation*}

Mit dieser Gleichung lässt sich nun $W$ berechnen. In der Praxis wird die obere Integrationsgrenze von $\infty$ in $T^*$ geändert, wobei
$j\left(T^*\right)\approx 0$ gilt. $T^*$ soll demnach groß genug sein, um eine Gleichverteilung der Dipole herbeizuführen. 