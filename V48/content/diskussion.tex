\section{Diskussion}
\label{sec:Diskussion}

Relative Fehler der Messergebnisse werden im Folgenden gemäß der Formel

\begin{equation}
    \Delta x = \left| \frac{x_\text{mess} - x_\text{theo}}{x_\text{theo}}\right|
\end{equation}

bestimmt. 

Von dem Theoriewert der Aktivierungsenergie $W_\text{theo} = \SI{0.66 +- 0.01}{\eV}$ \cite{Buch}
weichen die Ergebnisse der Approximationsmethode um

\begin{align*}
    \Delta W_1 &= \SI{18.18}{\percent} \\
    \Delta W_2 &= \SI{13.64}{\percent}
\end{align*}

und die der Integrationsmethode um

\begin{align*}
    \Delta W_1 &= \SI{136.82}{\percent} \\
    \Delta W_2 &= \SI{145.00}{\percent} 
\end{align*}

ab.

Es ist zu beobachten, dass die Ergebnisse der Integrationsmethode sehr viel stärker vom Theoriewert abweichen, 
als die der Näherungsmethode. Mögliche Fehlerquellen sind die numerische Natur der Integration, sowie Abweichungen der
Heizraten der einzelnen Messschritte von ihrem Mittelwert. Zudem ist die Approximation der Daten komplexer als bei der
Näherungsmethode. Aufgrund des sehr unsicher bestimmten Näherungsparameters $a$, konnten dort keine aussagekräftigen Fehlerangaben
gemacht werden. \\

Dies spiegelt sich auch in den Relaxationszeiten wieder, die mit den Werten der Aktivierungsenergie berechnet wurden.
Die nach der Integrationsmethode bestimmten Werte

\begin{align*}
    \tau_{0,1} &\approx \num{6.64} \cdot 10^{-32} \si{\second}\\
    \tau_{0,2} &\approx \num{7.55} \cdot 10^{-32} \si{\second} \; .
\end{align*}

weichen um achtzehn Größenornungen vom Theoriewert
$\tau_{0,\text{theo}} = \SI{4 +- 2}{\second}$ \cite{Buch} nach unten ab. Zum Teil kann dafür die hohe Abhängigkeit der Relaxationszeitberechnung
von den gemessenen Maximaltemperaturen $T_\text{max}$ Ursache sein.

Die nach der Näherungsmethode bestimmten Relaxationszeiten weichen nur um eine Größenordung ab.\\

Weiterhin ist zu beachten, dass in den gemessenen Spektren zwei Peaks vorkommen. Die höheren sind auf Relaxationsprozesse
höherer Ordnung durch Wechselwirkung nicht direkter Nachbarn der Ionen im Gitter zurückzuführen. 


