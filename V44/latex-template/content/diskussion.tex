\section{Diskussion}
\label{sec:Diskussion}

Zuerst lässt sich feststellen, dass die Theoriekurven des Wafers beide Kiessig-Oszillationen im Verlauf aufzeigen.
Sie gleichen sich im Verlauf stark, spalten sich jedoch für einen größer werdenden Wellenvektorübertrag $q$.
Die Reflektivität des Wafers mit theoretisch glatten Oberflächen ist dabei größer als mit relastisch rauen 
Oberflächen. Außerdem liegen beide Theoriekurven signifikant unter der Fresnelreflektivität einer 
einzelnen Siliziumoberfläche.
Diese Zusammenhänge sind physikalisch logisch. Glatte Oberflechen reflektieren besser als raue.
Für die nach verschiedenen Methoden bestimmten Geometriewinkel

\vspace{-10pt}
\begin{align*}
    \alpha_g &= \SI{0.74}{\degree} & \alpha'_g &= \SI{0.8}{\degree}
\end{align*}

ergibt sich eine relative Abweichung von $\SI{7.5}{\percent}$.
Die durch die Oszillationsminima abgeschätzte Schichtdicke

\vspace{-10pt}
\begin{equation*}
    d_\text{PS,est} = (\num{882.43 +- 0.04}) \cdot 10^{-10} \si{\meter} \; .
\end{equation*}

weicht um etwa $\SI{3.2}{\percent}$ von der durch die Theoriekurve bestimmte 
Schichtdicke 

\vspace{-10pt}
\begin{equation*}
    d_\text{PS,theo} = \SI{855e-10}{\meter}
\end{equation*} 

ab. Diese geringe Abweichung ist dadurch zu erklären, dass die Theoriekurve nicht
optimal an die Messdaten angenähert werden kann.

Die Korrekturterme konnten zu 

\vspace{-25pt}
\begin{align*}
    \delta_\text{PS} &= \num{0.7e-6} \; , \\
    \delta_\text{Si} &= \num{6.7e-6} \; .
\end{align*}

bestimmt werden und weichen um $\SI{80}{\percent}$ und 
$\SI{11.8}{\percent}$ von den Literaturwerten \cite{tolan}

\vspace{-15pt}
\begin{align*}
    \delta_\text{PS} &= \num{0.7e-6} \; , \\
    \delta_\text{Si} &= \num{6.7e-6} \;
\end{align*}

ab. Diese Abweichungen ergeben sich möglicherweise aus einer 
ungenauen Bestimmung der Parameter.
Die kritischen Winkel 

\vspace{-20pt}
\begin{align*}
    \alpha_\text{c,PS} &= \SI{0.068}{\degree} \; , \\
    \alpha_\text{c,Si} &= \SI{0.210}{\degree}
\end{align*}
 
weichen um  $\SI{54.7}{\percent}$ und 
$\SI{4.5}{\percent}$ von den Literaturwerten \cite{tolan}

\vspace{-20pt}
\begin{align*}
    \alpha_\text{c,PS,lit} &= \SI{0.15}{\degree} \; , \\
    \alpha_\text{c,Si,lit} &= \SI{0.22}{\degree}
\end{align*}

ab. Hier liegt die Abweichung für Polystyrol erneut deutlich höher
als für Silizium. Insgesamt halten sich die Abweichungen des Versuches allerdings im Rahmen
und er liefert nah an der Theorie liegende Werte.

\section{Anhang}

\begin{lstlisting}[language=Python]
    \caption{Programmcode für zweiten Auswertungsteil}

    import numpy as np
    import matplotlib.pyplot as plt
    from scipy import optimize
    from scipy.optimize import curve_fit
    from scipy import signal as sig
    from uncertainties import ufloat
    import uncertainties.unumpy as unp
    
    #Versuchsgrößen
    l = 1.54e-10 # Wellenlänge
    ai = np.pi/180 * np.arange(6e-2, 1.505, 5e-4) 
    #Winkel -> x−Werte der Theoriekurve
    k = 2*np.pi / l #Wellenvektor
    qz = 2*k * np.sin(ai) 
    #Wellenvektorübertrag -> y-Werte der Theoriekurve
    
    #Parameter des Parratt-Algorithmus
    
    #Brechungsindizes
    d1 = 0.7e-6 #Polysterol -> Amplitude vergrößert + negativer Offset
    d2 = 6.7e-6 #Silizium -> Amplitude vergkleinert + positiver Offset
    #
    n1 = 1 #Luft
    n2 = 1 - d1 #Polysterol
    n3 = 1 - d2 #Silizium
    
    #Rauigkeit
    s1 = 7.9e-10 
    #Polysterol -> Amplitude verkleinert bei hinteren Oszillationen
    s2 = 5.7e-10 
    #Silizium -> Senkung des Kurvenendes und 
    #Amplitudenverkleinerung der Oszillationen
    
    #Schichtdicke
    z = 855e-10 # verkleinert Oszillationswellenlänge
    
    #####################################################################
    
    #Messdaten
    x, y = np.genfromtxt('data/reflektivitätsscan.UXD', unpack=True)
    y = y / (5 * 1636650) 
    #Normierung auf die 5fache maximale Intensität des Z-Scans
    
    
    def plotten(phi, I):
        fig, ax1 = plt.subplots()
        #
        q = 2*k * np.sin(np.pi/180 * phi) #Winkel -> Wellenvektorübertrag
        phi_d, I_d = np.genfromtxt('data/difscan.UXD', unpack=True) 
        #Messdaten des diffusen Scans
        I_d = I_d / 1636650 #Normierung
        q_d = 2*k * np.sin(np.pi/180 * phi_d) 
        #
        I_d_k = diff_k(I, I_d) #Korrektur der Messdaten um Diffusen Scan
        I_res = geofaktor() #Korrektur der Messdaten um Geometriefaktor
        #
        q_min , I_min , peak_array = min(I_res, q) 
        #findet Oszillationsminima
        d_layer = layer(q, peak_array) #Bestimmt Schichtdicke
        print(d_layer)
        #
        u = 40 #Ende des Plateaus
        beg = 11 # Anfang sinnvoller Messdaten
        end = 300 # Ende sinnvoller Messdaten
        #
        unkorr = I[beg:end] * 1e-1 
        #Übersichtshalber werden die Daten nach unten verschoben
        diff_data = I_d[beg:end] * 1e-1
        #Plots
        ax1.plot(q[beg:end], unkorr, linestyle='dashed', linewidth=0.7, 
        color='#FB9D02', label='Messdaten x 0,1')

        ax1.plot(q_d[beg:end], diff_data, linestyle='dashed', 
        linewidth=0.7, color='#4DD30A', label='Diffuse Daten x 0,1')
        ax1.plot(q_d[beg:end], I_d_k[beg:end], linewidth=0.7, 
        color='#AD0EBD', label='Daten um diffuse Daten korrigiert x 0,1')
        ax1.plot(q[beg:end], I_res[beg:end], linewidth=0.7, 
        color='#045DF9', label='Daten um Geometriefaktor korrigiert')
        ax1.plot(q_min, I_min, 'x', color='#06189A', 
        label='Schichtdickenminima')
        #
        phicrit_PS , qcrit_PS = phi_crit(d1) 
        #Bestimmung der kritischen Winkel
        ax1.axvline(x=qcrit_PS, ymin=0, ymax=1, linestyle='dashed', 
        linewidth = 0.7, color='#ABB2B9', label=r'$\alpha_c$ für PS') 
        #Vertikales Plotten des Winkels

        phicrit_Si , qcrit_Si = phi_crit(d2)
        ax1.axvline(x=qcrit_Si, ymin=0, ymax=1, linestyle='dashed', 
        linewidth = 0.7, color='#566573', label=r'$\alpha_c$ für Si')
        #
        par = parratt(z) #Aufruf des Parrat-Algorithmus
        par2, r13 = parratt2(z)
        plt.plot(qz, par, linewidth=0.7, color='#F90421',
        label='Theoriekurve (raues Si)')
        plt.plot(qz, par2, linewidth=0.7, color='#FA58AC',
        label='Theoriekurve (glattes Si)', alpha=0.5)
        plt.plot(qz, r13, linewidth=0.7, color='#FA58AC',
        linestyle='dashed', label='Fresnelrefektivität Si', alpha=0.5)
        #
        #
        plt.yscale('log')
        plt.grid(alpha = 0.2)
        ax1.legend(fancybox=True, ncol = 1, loc='upper right')
        ax1.set_xlabel('q (1/m)')
        ax1.set_ylabel('Reflektivität')
        fig.tight_layout()
        plt.savefig('plot4.pdf')
    
    
    def diff_k(I, I_d) :
        I_d_k = np.zeros(np.size(I))
        for i in np.arange(np.size(I)):
            I_d_k[i] = 1e-1 * (I[i] - I_d[i])
        return I_d_k
    
    
    def geofaktor():
        ag = 0.74
        I_res = np.zeros(np.size(x))
        for i in np.arange(np.size(x)):
            if(x[i] <= ag and x[i] > 0):
                I_res[i] = y[i] * np.sin(ag) / np.sin(x[i])
            else :
                I_res[i] = y[i]
        return I_res
    
    
    def min(I_res, q):
        I_log = np.zeros(np.size(I_res)) #logarithmierte Daten
        for i in np.arange(np.size(I_res)):
            if (I_res[i] != 0):
                I_log[i] = np.log10(I_res[i])
            else :
                I_log[i] = 0
        #
        I_log_n = -1 * I_log #Negatives Vorzeichen 
        #-> Minima werden zu Maxima -> Zur Arbeit mit find_peaks
        peakfinder = sig.find_peaks(I_log_n, prominence = 7e-2)
        #
        peak_array = [78, 88, 98, 108, 118, 128, 139, 149, 
        160, 170, 181, 192, 203, 226] #... Manuell angepasst
        #
        peak_q = []
        peak_I = []
        #
        for i in peak_array:
            peak_q.append(q[i])
            peak_I.append(I_res[i])
        #
        return peak_q , peak_I , peak_array
    
    
    def layer(q, peak_array):
        dist_peaks = [] #Peakabstand
        for i in peak_array:
            if(i < 277):
                dist_peaks.append(q[i+1] - q[i])
            else:
                pass
        mean = np.mean(dist_peaks)
        std = np.std(dist_peaks)
        d = 2*np.pi / ufloat(mean, std)
        return d
    
    
    def phi_crit(delta) :
        phicrit = 180/np.pi * np.sqrt(2 * delta)
        qcrit = 2*k * np.sin(np.pi/180 * phicrit)
        return (phicrit, qcrit)
    
    
    def parratt(z):
        kz1 = k * np.sqrt(np.abs(n1**2 - np.cos(ai)**2))
        kz2 = k * np.sqrt(np.abs(n2**2 - np.cos(ai)**2))
        kz3 = k * np.sqrt(np.abs(n3**2 - np.cos(ai)**2))
        #
        r12 = (kz1 - kz2) / (kz1 + kz2) * np.exp(-2 * kz1 * kz2 * s1**2)
        r23 = (kz2 - kz3) / (kz2 + kz3) * np.exp(-2 * kz2 * kz3 * s2**2)
        #
        x2 = np.exp(0 - (kz2 * z) * 2j) * r23
        x1 = (r12 + x2) / (1 + r12 * x2)
        par = np.abs(x1)**2
        #Strecke vor Beginn der Oszillationen auf 1 setzen
        for i in np.arange(np.size(par)):
            if (i <= 296): #296 manuell angepasst
                par[i] = 1
            else:
                pass
        return par
    
    def parratt2(z):
        kz1 = k * np.sqrt(np.abs(n1**2 - np.cos(ai)**2))
        kz2 = k * np.sqrt(np.abs(n2**2 - np.cos(ai)**2))
        kz3 = k * np.sqrt(np.abs(n3**2 - np.cos(ai)**2))
        #
        r12 = (kz1 - kz2) / (kz1 + kz2)
        r23 = (kz2 - kz3) / (kz2 + kz3)
        r13 = (kz1 - kz3) / (kz1 + kz3)
        #
        x2 = np.exp(0 - (kz2 * z) * 2j) * r23
        x1 = (r12 + x2) / (1 + r12 * x2)
        par = np.abs(x1)**2
        #Strecke vor Beginn der Oszillationen auf 1 setzen
        for i in np.arange(np.size(par)):
            if (i <= 296): #296 manuell angepasst
                par[i] = 1
                r13[i] = 1
            else:
                pass
        return par, r13

    plotten(x, y)
\end{lstlisting}