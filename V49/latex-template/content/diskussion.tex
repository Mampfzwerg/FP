\section{Diskussion}
\label{sec:Diskussion}


Eine Fehlerquelle, die sich auf alle Versuchsteile aufwirkt, ist die große Änderung der Temperatur während des Versuches,
wodurch das Magnetfeld nicht stabil werden konnte und sich Ablesefehler in der Justierung nicht vermeiden ließen. 
Des Weiteren können Verunreinigungen auf der Außenseite der Probe oder innerhalb dieser trotz einer erfolgten Reinigung 
nicht ausgeschlossen werden.\\
Die Relaxationszeiten ergaben sich zu 

\begin{align*}
    T_1 &= \SI{1.03(8)}{\second},\\
    T_2 &= \SI{4(8)}{\second}.
\end{align*}

Hier ist $T_2 > T_2$, was der Theorievorhersage widerspricht. Allerdings ist der Fehler von $T_2$ sehr groß, was darauf 
zurückzuführen ist, dass nur wenige Peaks des augenommenen Signals betrachtet werden konnten. \\
Ein Vergleich mit den Literaturwerten \cite{diff}

\begin{align*}
    T_\text{1,lit} &= \SI{3.09(15)}{\second},\\
    T_\text{2,lit} &= \SI{1.52(9)}{\second},
\end{align*}

liefert die Abweichungen $\SI{66.6}{\percent}$ für $T_1$ und $\SI{163.2}{\percent}$ für $T_2$.\\
Für die Diffusionskonstante ergab sich der Wert 

\begin{equation*}
    D = \SI{1.93(5)e-9}{\metre^2\per\second}.
\end{equation*}

Der Vergleich mit dem Literaturwert \cite{diff}

\begin{equation*}
    D_\text{lit} = \SI{2.78(4)e-9}{\metre^2\per\second}
\end{equation*}

liefert eine Abweichung von $\SI{30.6}{\percent}$.\\
Abschließend wird der Molekülradius von Wasser betrachtet: \cite{radius} 

\begin{align*}
    r_\text{Exp} &= \SI{1.258(33)e-10}{\metre},\\
    r_\text{hcp} &= \SI{1.74e-10}{\metre},\\
    r_\text{lit} &= \SI{1.69e-10}{\metre}.
\end{align*}

Dabei weicht $r_\text{hcp}$ um $\SI{3}{\percent}$ von $r_\text{lit}$ ab, woraus sich schliesen lässt, dass die hcp Struktur eine gute 
Annahme darstellt. Die Abweichung von $r_\text{Exp}$ beträgt $\SI{25.5}{\percent}$. Diese Abweichung liegt vermutlich an der großen 
Temperaturveränderung während des Versuchs. Außerdem ist erkennbar, dass $D$ im Experiment nicht perfekt einer $\tau^3$-Abhängigkeit 
folgt. Dies deutet auf Messfehler von $D$ hin, was sich auch auf die Bestimmung von $r$ auswirkt. 
