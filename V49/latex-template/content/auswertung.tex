\section{Auswertung}
\label{sec:Auswertung}

Zunächst wird eine Justage für eine möglichst große Amplitude des FID und eine Minimierung des Imaginärteils durchgeführt. 
Diese führt zu folgenden Werten

\begin{align}
  \omega_\text{L} &= \SI{0}{\mega\hertz},\\
  \phi &= \SI{-0}{\degree}
\end{align}

für die Lamorfrequenz $\omega_\text{L}$ und die Phase $\phi$. Die Pulslängen des $\SI{90}{\degree}$ und des $\SI{180}{\degree}$
ergeben sich zu 

\begin{align}
  \symup{\Delta} t_{90} &= \SI{0}{\micro\second},\\
  \symup{\Delta} t_{180} &= \SI{0}{\micro\second}.
\end{align}

Die Temperaturmessungen zu Beginn der Messungen und als Abschluss dieser, ergeben als Temperaturen innerhalb des Magneten: 

\begin{align}
  T_\text{Beginn} &= \SI{0}{\celsius},\\
  T_\text{Ende} &= \SI{0}{\celsius}.
\end{align}

\subsection{Bestimmung der Spin-Gitter Relaxationszeit}

In diesem Abschnitt soll die Spin-Gitter Relaxationszeit $T_1$ bestimmt werden. Hierzu wurde die Spannungsamplitude der 
induzierten Spannung für verschiedene Pulsabstände $\tau$ gemessen. Die Messwerte sind in Tabelle \ref{tab:mess1} zu finden.

\begin{table}
  \centering
  \caption{Messwerte zur Bestimmung der Relaxationszeit $T_1$. Es wurde die Spannungsamplitude $U$ für verschiedene Pulsabstände $\tau$ gemessen.}
  \label{tab:mess1}
  \sisetup{table-format=2.1}
  \begin{tabular}{c c c c}
  \toprule
  $\tau \,/\, \si{\milli\second}$ & $U \,/\, \si{\volt}$ & $\tau \,/\, \si{\volt}$
  & $U \,/\, \si{\volt}$\\
  \midrule 
      0 & 0 & 0 & 0\\
      0 & 0 & 0 & 0\\
      0 & 0 & 0 & 0\\
      0 & 0 & 0 & 0\\
      0 & 0 & 0 & 0\\
      0 & 0 & 0 & 0\\
      0 & 0 & 0 & 0\\
      0 & 0 & 0 & 0\\
      0 & 0 & 0 & 0\\
      0 & 0 & 0 & 0\\
  \bottomrule
  \end{tabular}
\end{table}

Diese Messwerte wurden graphisch dargestellt in Abbildung ... . Mit der Gleichung \eqref{eqn:SGR} wird eine Ausgleichsrechnung
anhand von 

\begin{equation*}
  U\left(\tau\right) = U_0 \left(1-2\exp{\left(-\frac{\tau}{T_1}\right)}\right) + U_1
\end{equation*}

durchgeführt. Dabei dient $U_1$ zum Ausgleich der Nulllinie. 
Die Parameter der Ausgleichsrechnung ergeben sich mittels \textit{python} zu:

\begin{align}
  U_0 &= \SI{0(0)}{\volt},\\
  T_1 &= \SI{0(0)}{\milli\second},\\
  U_1 &= \SI{0(0)}{\volt}.
\end{align}

\subsection{Bestimmung der Spin-Spin Relaxationszeit}

Die Bestimmung der Relaxationszeit $T_2$ erfolgt mithilfe der Meiboom-Gill-Methode. Anschließend wird die Carr-Purcell-Methode 
betrachtet, die allerdings kein Ergebnis für $T_2$ liefern kann.

\subsubsection{Meiboom-Gill-Methode}

Das von der Meiboom-Gill Methode gelieferte Signal ist in Abbildung ... zu sehen. Mithilfe der Funktion \textit{scipy.signal.find\_peaks}\cite{scipy} 
ZITIEREN werden die Peaks mit positiver Spannungsamplitude $U$ aus dem Signal herausgefiltert. Die Bestimmung der Peaks erfolgt 
dabei über die CSV-Datei des Signals. Die so erhaltenen Messwerte sind in Tabelle \ref{tab:mess2} aufgeführt und in Abbildung ... 
graphisch dargestellt. 

\begin{table}
  \centering
  \caption{Herausgefilterte Peaks des Signals zur Bestimmung von $T_2$.}
  \label{tab:mess2}
  \sisetup{table-format=2.1}
  \begin{tabular}{c c c c}
  \toprule
  $t \,/\, \si{second}$ & $U \,/\, \si{\volt}$ & $t \,/\, \si{\second}$
  & $U \,/\, \si{\volt}$\\
  \midrule 
      0 & 0 & 0 & 0\\
      0 & 0 & 0 & 0\\
      0 & 0 & 0 & 0\\
      0 & 0 & 0 & 0\\
      0 & 0 & 0 & 0\\
      0 & 0 & 0 & 0\\
      0 & 0 & 0 & 0\\
      0 & 0 & 0 & 0\\
      0 & 0 & 0 & 0\\
      0 & 0 & 0 & 0\\
  \bottomrule
  \end{tabular}
\end{table}

Mit der Gleichung \eqref{eqn:MGM} wird eine Ausgleichsrechnung anhand von 

\begin{equation*}
  U\left(t\right) = U_0 \exp{\left(-\frac{t}{T_2}\right)} + U_2
\end{equation*}

durchgeführt, wobei $t = 2\tau$ gilt und $U_2$ die Verschiebung der Nulllinie ist. 

Die Parameter der Ausgleichsrechnung ergeben sich mittels \textit{python} zu:

\begin{align}
  U_0 &= \SI{0(0)}{\volt},\\
  T_2 &= \SI{0(0)}{\second},\\
  U_2 &= \SI{0(0)}{\volt}.
\end{align}

\subsubsection{Carr-Purcell-Methode}

Um genaue Ergebnisse zu erhalten, ist eine exakte Einstellung der Pulslänge für den $\SI{180}{\degree}$-Puls notwendig. Dies ist 
in der Praxis nicht umsetzbar und würde zu großen Fehlern führen. Es ergibt sich aus dieser Methode demnach lediglich ein Signal,
das in Abbildung .. zu sehen ist. $T_2$ kann daraus nicht bestimmt werden.  

\subsection{Bestimmung der Magnetfeldgradientenstärke}

Zur Bestimmung der Gradientenstärke $G$ wird das gut sichbare Echo in Abbildung ... fouriertransformiert.

Die Fouriertransformation erfolgt mittels \textit{numpy}\cite{numpy} und ergibt ein Spektrum, welches in Abbildung ... zu sehen ist. 

Mit dem Durchmesser des Proberöhrchens $d = \SI{4.2}{\milli\metre}$, dem gyromagnetischem Verhältnis für Protonen 
$\gamma = \SI{2.67e8}{\per\second\tesla}$ und dem Durchmesser des Halbkreises im Spektrum des Echos $d_\text{f} \approx \SI{0}{\hertz}$
ergibt sich $G$ zu 

\begin{equation*}
  G = \frac{2\pi d_\text{f}}{\gamma d} = \SI{0}{\tesla\per\metre}.
\end{equation*}

\subsection{Bestimmung der Diffusionskonstante}

Zur Bestimmung der Diffusionskonstante $D$ werden die gemessenen Spannungsamplituden $U$ für verschiedene Pulsabstände $\tau$ aus 
Tabelle... in Abbildung ... graphisch dargestellt.  

\begin{table}
  \centering
  \caption{Gemessene Spannungsamplituden $U\left(\tau\right)$ zur Bestimmung der Diffusionskonstante $D$.}
  \label{tab:mess3}
  \sisetup{table-format=2.1}
  \begin{tabular}{c c c c}
  \toprule
  $\tau \,/\, \si{\milli\second}$ & $U \,/\, \si{\volt}$ & $\tau \,/\, \si{\milli\second}$
  & $U \,/\, \si{\volt}$\\
  \midrule 
      0 & 0 & 0 & 0\\
      0 & 0 & 0 & 0\\
      0 & 0 & 0 & 0\\
      0 & 0 & 0 & 0\\
      0 & 0 & 0 & 0\\
      0 & 0 & 0 & 0\\
      0 & 0 & 0 & 0\\
      0 & 0 & 0 & 0\\
      0 & 0 & 0 & 0\\
      0 & 0 & 0 & 0\\
  \bottomrule
  \end{tabular}
\end{table}

Mit der Gleichung \eqref{eqn:DK} wird eine Ausgleichsrechnung anhand von 

\begin{equation*}
  U\left(2\tau\right) = U_0 \exp{\left(-\frac{-2\tau}{T_2}\right)} \exp{\left(-\frac{2}{3}D\gamma^2G^2\tau^3\right)} + U_3
\end{equation*}

durchgeführt. 




Die Parameter der Ausgleichsrechnung ergeben sich mittels \textit{python} zu:

\begin{align}
  U_0 &= \SI{0(0)}{\volt},\\
  D &= \SI{0(0)}{\metre^2\per\second},\\
  U_3 &= \SI{0(0)}{\volt}.
\end{align}

Um die $\tau^3$-Abhängigkeit zu überprüfen kann noch eine lineare Regression mit 

\begin{equation}
  \ln{\left(U\left(\tau\right)\right)}-\frac{2\tau}{T_2} = a\tau^3+b
\end{equation}

durchgeführt werden. Dies ist in Abbildung ... zu sehen. Die Regressionsparameter ergeben sich zu 

\begin{align*}
  a &= \SI{0(0)}{\per\second^3},\\
  b &= \num{0(0)},
\end{align*}

womit die $\tau^3$-Abhängigkeit in etwa bestätigt werden kann\dots

\subsection{Bestimmung des Molekülradius}

Der Molekülradius kann mithilfe der Stokes-Formel bestimmt werden:

\begin{equation*}
  D = \frac{k_\text{B}T}{6\pi\eta r} \rightleftarrows r = \frac{k_\text{B}T}{6 \pi\eta D}.
\end{equation*}

Mit $T = \SI{0}{\kelvin}$ und einer Viskosität $\eta = \SI{890.2}{\micro\pascal\second}$\cite{vis} ergibt sich ein Molekülradius
von

\begin{equation}
  r = \SI{0(0)}{\metre}.
\end{equation}

Zur Berechnung eines Vergleichwertes wird angenommen, dass die Molküle in einer hexagonal dichtesten Kugelpackung angeordnet 
sind. Die Raumfüllung beträgt dort $\SI{74}{\percent}$. Mit einer Molekülmasse von $m = \frac{M_\text{mol}}{N_\text{A}} = \SI{2.99e-26}{\kilo\gram}$
und einer Dichte von $\rho=\SI{995}{\kilo\gram\per\metre^3}$\cite{rho} ergibt sich

\begin{equation}
  r_\text{hcp} = \left(\frac{m\cdot 0.74}{\frac{4}{3}\pi\rho}\right)^{\frac{1}{3}} = \SI{1.74e-10}{\metre}.
\end{equation}


%\begin{figure}
%  \centering
%  \includegraphics{plot.pdf}
%  \caption{Plot.}
%  \label{fig:plot}
%\end{figure}
