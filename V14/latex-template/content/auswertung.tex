\section{Auswertung}
\label{sec:Auswertung}

Als erster Schritt wird die am Szintillationsdetektor ankommende Strahlung ohne ein Objekt im Strahlengang gemessen. \\
Zu sehen ist das nach Abbildung \ref{fig:würfel} erwartete Spektrum. Dieses lässt sich in zwei charakteristische Bereiche teilen: Das 
Compton-Kontinuum und der Photopeak. Im Compton Kontinuum lässt sich schwach ein Peak feststellen, der von der Rückstrahlung 
stammt. Die Compton-Kante ist klar zu sehen und tritt bei Energien von etwa $\SI{480}{\kilo\electronvolt}$ auf.\\
Es wurde hier und in allen folgenden Versuchsteilen eine Zeit von $\symup{\Delta}t = \SI{300}{\second}$ gemessen, um die 
statistischen Unsicherheiten bei $< \SI{3}{\percent}$ zu halten. Die Counts 
ergeben sich dabei zu

\begin{equation*}
  C = 6743 \pm 82.
\end{equation*}

Diese bezeichnen, die durch den Szintillationsdetektor ausgelösten Pulse. 
Mit diesen Messwerten ergibt sich die Eingangsintensität gemäß 

\begin{equation}
  I_0 = \frac{C}{\symup{\Delta}t} = \SI{22.48+-0.27}{\per\second}.
  \label{eqn:Int}
\end{equation}

Der Fehler berechnet sich dabei nach

\begin{equation*}
  \sigma_{I_0} = \sqrt{\left(\frac{\sigma_\text{C}}{C}\right)^2}I_0.
\end{equation*}

\subsection{Untersuchung des Aluminiumgehäuses}

Die zu untersuchenden Proben sind von einem Aluminiumgehäuse umgeben, welches unerwünschte Absorptionen verursacht. Um diese
erkennen zu können, wird ein Würfel bestehend aus dem Aluminiumgehäuse untersucht. Die aufgenommenen Messwerte sind in 
Tabelle \ref{tab:mess1} dargestellt. 

\begin{table}[H]
  \centering
  \caption{Messwerte für das Aluminiumgehäuse.}
  \label{tab:mess1}
  \sisetup{table-format=2.1}
  \begin{tabular}{c c c}
  \toprule
  $\text{Counts} \; C$ & $\symup{\Delta} t \;/\; \si{\second}$ & $\text{Projektionstyp} \;/\; \si{\per\second}$ \\
  \midrule
      $\num{6665+-82}$ & 300 & $I_{0,5}$\\
      $\num{6503+-81}$ & 300 & $I_{0,2}$\\
      $\num{6375+-80}$ & 300 & $I_{0,8}$\\
  \bottomrule
  \end{tabular}
\end{table}

Diese können als Nullmessung des jeweiligen Projektionstyps angenommen werden. Gemäß Gleichung \eqref{eqn:Int} und deren Fehler, ergeben
sich die Zählraten zu: 

\begin{align*}
  I_{0,5} &= \SI{22.2+-0.3}{\per\second},\\ 
  I_{0,2} &= \SI{21.7+-0.3}{\per\second},\\ 
  I_{0,8} &= \SI{21.3+-0.3}{\per\second}.   
\end{align*}

\subsection{Bestimmung des Materials eines homogenen Würfels}

Die untersuchten Projektionen sind gemäß Abbildung \ref{fig:würfel} $I_5$, $I_6$, $I_7$ und $I_8$. Die gemessenen Werte sind in 
Tabelle \ref{tab:mess2} aufgeführt. 

\begin{table}[H]
  \centering
  \caption{Messwerte für den Würfel zwei.}
  \label{tab:mess2}
  \sisetup{table-format=2.1}
  \begin{tabular}{c c c c}
  \toprule
  $\text{Counts} \; C$ & $\symup{\Delta} t \;/\; \si{\second}$ & $\text{Projektionstyp}$ & $\text{Zählrate} \;/\; \si{\per\second}$\\
  \midrule
      $\num{1063+-33}$ & 300 & $I_{5}$ & $\num{3.5+-0.1}$\\
      $\num{1090+-33}$ & 300 & $I_{6}$ & $\num{3.6+-0.1}$\\
      $\num{1547+-39}$ & 300 & $I_{7}$ & $\num{5.2+-0.1}$\\
      $\num{674+-26}$ & 300 & $I_{8}$  & $\num{2.3+-0.1}$\\
  \bottomrule
  \end{tabular}
\end{table}

Aus diesen Werten ergeben sich der Absorptionskoeffizienten gemäß 

\begin{equation*}
  \mu_i = \frac{1}{l_i}\ln{\left(\frac{I_{0,j}}{I_i}\right)}
\end{equation*}

Dabei bezeichnet $l_i$ die zurückgelegte Strecke der Gammastrahlung durch den Würfel und $I_0,j$ die entsprechende Zählrate. Dieser 
Zusammenhang führt zu den Werten in Tabelle \ref{tab:mess3}.

\begin{table}[H]
  \centering
  \caption{Berechnete Werte der Absorptionskoeffizienten für Würfel zwei.}
  \label{tab:mess3}
  \sisetup{table-format=2.1}
  \begin{tabular}{c c}
  \toprule
  $\text{Projektion}$ & $\mu \;/\; \si{\per\centi\metre}$\\
  \midrule
      5 & $\num{0.612+-0.011}$\\
      6 & $\num{0.604+-0.011}$\\
      7 & $\num{0.508+-0.010}$\\
      8 & $\num{0.534+-0.010}$\\
  \bottomrule
  \end{tabular}
\end{table}

Dabei ergibt sich der Fehler mithilfe der Gaußschen Fehlerfortpflanzung:

\begin{equation*}
  \symup{\Delta} \mu_i = \sqrt{\left(-\frac{\symup{\Delta}I_i}{l_i I_i}\right)^2\left(\frac{\symup{\Delta C_{0,j}}}{l_j C_{0,j}}\right)^2}.
\end{equation*}

Durch Bildung des Mittelwertes ergibt sich der Absorptionskoeffizient zu 

\begin{equation*}
  \bar{\mu_2} = \SI{0.564+-0.044}{\per\centi\metre}. 
\end{equation*}

Bei einem Vergleich mit den Literaturwerten ergibt sich, dass der Würfel mit einer Abweichung von $\SI{0.9}{\percent}$ aus Eisen zu bestehen scheint. 

\subsection{Bestimmung des Materials eines homogenen weiteren Würfels}

Die untersuchten Projektionen sind erneut $I_5$, $I_6$, $I_7$ und $I_8$. Die gemessenen Werte sind in 
Tabelle \ref{tab:mess4} aufgeführt. 

\begin{table}[H]
  \centering
  \caption{Messwerte für den Würfel drei.}
  \label{tab:mess4}
  \sisetup{table-format=2.1}
  \begin{tabular}{c c c c}
  \toprule
  $\text{Counts} \; C$ & $\symup{\Delta} t \;/\; \si{\second}$ & $\text{Projektionstyp}$ & $\text{Zählrate} \;/\; \si{\per\second}$\\
  \midrule
      $\num{4849+-70}$ & 300 & $I_{5}$ & $\num{16.16+-0.2}$\\
      $\num{4860+-70}$ & 300 & $I_{6}$ & $\num{16.20+-0.2}$\\
      $\num{4738+-69}$ & 300 & $I_{7}$ & $\num{15.79+-0.2}$\\
      $\num{4338+-66}$ & 300 & $I_{8}$  & $\num{14.46+-0.2}$\\
  \bottomrule
  \end{tabular}
\end{table}

Aus diesen Werten ergeben sich erneut die Absorptionskoeffizienten in Tabelle \ref{tab:mess5}.

\begin{table}[H]
  \centering
  \caption{Berechnete Werte der Absorptionskoeffizienten für Würfel drei.}
  \label{tab:mess5}
  \sisetup{table-format=2.1}
  \begin{tabular}{c c}
  \toprule
  $\text{Projektion}$ & $\mu \;/\; \si{\per\centi\metre}$\\
  \midrule
      5 & $\num{0.106+-0.006}$\\
      6 & $\num{0.105+-0.006}$\\
      7 & $\num{0.112+-0.007}$\\
      8 & $\num{0.095+-0.005}$\\
  \bottomrule
  \end{tabular}
\end{table}

Durch Bildung des Mittelwertes ergibt sich der Absorptionskoeffizient zu 

\begin{equation*}
  \bar{\mu_3} = \SI{0.105+-0.006}{\per\centi\metre}. 
\end{equation*}

Bei einem Vergleich mit den Literaturwerten ergibt sich, dass der Würfel mit einer Abweichung von 
$\SI{9.5}{\percent}$ aus CH20 also Delrin zu bestehen scheint. 

\subsection{Bestimmung der Materialien in einem zusammengesetzten Würfel}

Der vierte Würfel ist aus unterschiedlichen Materialien zusammengesetzt. Die Messwerte und die Zählraten für die Projektionen 
sind in Tabelle \ref{tab:mess6} aufgeführt.

\begin{table}[H]
  \centering
  \small
  \caption{Messwerte für den Würfel vier.}
  \label{tab:mess6}
  \sisetup{table-format=2.1}
  \begin{tabular}{c c c c}
  \toprule
  $\text{Counts} \; C$ & $\symup{\Delta} t \;/\; \si{\second}$ & $\text{Projektion}$ & $\text{Zählrate} \;/\; \si{\per\second}$\\
  \midrule
      $\num{3949+-63}$ & 300 & 1 & $\num{13.16+-0.2}$\\
      $\num{1653+-41}$ & 300 & 2 & $\num{5.51+-0.1}$\\
      $\num{3641+-60}$ & 300 & 3 & $\num{12.14+-0.2}$\\
      $\num{2835+-53}$ & 300 & 4 & $\num{9.45+-0.2}$\\
      $\num{2799+-53}$ & 300 & 5 & $\num{9.33+-0.2}$\\
      $\num{2735+-53}$ & 300 & 6 & $\num{9.12+-0.2}$\\
      $\num{3505+-59}$ & 300 & 7 & $\num{11.68+-0.2}$\\
      $\num{791+-28}$ & 300 & 8 & $\num{2.64+-0.1}$\\
      $\num{2190+-47}$ & 300 & 9 & $\num{7.30+-0.2}$\\
      $\num{1493+-39}$ & 300 & 10 & $\num{4.98+-0.1}$\\
      $\num{1554+-39}$ & 300 & 11 & $\num{5.18+-0.1}$\\
      $\num{1678+-41}$ & 300 & 12 & $\num{5.59+-0.1}$\\
  \bottomrule
  \end{tabular}
\end{table}

Dann ergibt sich der Vektor der Absorptionskoeffizienten nach Gleichung \eqref{eqn:Abs} zu 

\begin{equation*}
  \vec{\mu} = 
    \begin{pmatrix}
      \mu_1\\
      \mu_2\\
      \mu_3\\
      \mu_4\\
      \mu_5\\
      \mu_6\\
      \mu_7\\
      \mu_8\\
      \mu_9\\
    \end{pmatrix}
    = 
    \begin{pmatrix}
      \num{0.514+-0.015}\\
      \num{0.178+-0.010}\\
      \num{0.381+-0.015}\\
      \num{0.337+-0.010}\\
      \num{0.482+-0.013}\\
      \num{0.199+-0.010}\\
      \num{0.174+-0.015}\\
      \num{0.373+-0.010}\\
      \num{0.562+-0.016}\\
    \end{pmatrix}
    \si{\per\centi\metre}
\end{equation*}

Der Fehler berechnet sich dabei für $\tilde{A} = \left(A^T \cdot A\right)^{-1}A^T$ wie folgt:

\begin{equation*}
  \sigma_{\mu,i} = \left(\sqrt{\frac{\tilde{A}^2\sigma_\text{C}^2}{C^2}+\frac{\tilde{A}^2\sigma_\text{I}^2}{I_0^2}}\right).
\end{equation*}

Zum Vergleich werden Literaturwerte benötigt, die sich in Tabelle \ref{tab:lit} finden lassen. 

\begin{table}[H]
  \centering
  \caption{Absorptionskoeffizienten $\mu$ in verschiedenenen Materialien \cite{lit}.}
  \label{tab:lit}
  \sisetup{table-format=2.1}
  \begin{tabular}{c c}
  \toprule
  $\text{Material}$ & $\mu \;/\; \si{\per\centi\metre}$\\
  \midrule
      Al & 0,201\\
      Pb & 1,175\\
      Fe & 0,569\\
      Messing & 0,605\\
      CH20  & 0,116\\
  \bottomrule
  \end{tabular}
\end{table}

Damit können die Materialien der einzelnen Würfel in Tabelle \ref{tab:mat} bestimmt werden. 

\begin{table}[H]
  \centering
  \caption{Identifizierung der Materialien.}
  \label{tab:mat}
  \sisetup{table-format=2.1}
  \begin{tabular}{c c c}
  \toprule
  $\text{Würfel}$ & $\text{Material}$ & $\text{Prozentuale Abweichung}$\\
  \midrule
    1 & Fe & 1,1\\
    2 & CH20 & 53,5\\
    3 & Fe & 33\\
    4 & CH20 & 190,5\\
    5 & Fe & 15,5\\
    6 & CH20 & 71,5\\
    7 & CH20 & 50\\
    8 & Fe & 34,5 \\
    9 & Fe & 1,2\\
  \bottomrule
  \end{tabular}
\end{table}