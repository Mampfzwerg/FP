\section{Diskussion}
\label{sec:Diskussion}

Die gemessenen Zählraten für die Nullmessung mit einem Aluminiummantel ergeben sich zu 

\vspace{-25pt}
\begin{align*}
    I_{0,5} &= \SI{22.2+-0.3}{\per\second},\\ 
    I_{0,2} &= \SI{21.7+-0.3}{\per\second},\\ 
    I_{0,8} &= \SI{21.3+-0.3}{\per\second}.   
\end{align*}

Es ist ein sehr kleiner Unterschied zwischen den Nullmessungen zu erkennen. Es sind kleine Unterschiede zu erwarten. Vor allem 
die Werte für $I_{0,2}$ und $I_{0,8}$ liegen sehr nahe beieinander, was zu erwarten ist, weil beide die Diagonalen der Würfel kennzeichnen. 
Die Nullmessung scheint demnach gut funktioniert zu haben.\\

Für den Absorptionskoeffizient des ersten homogenen Würfeld ergibt sich unter Verwendung der Werte der Nullmessung 

\vspace{-5pt}
\begin{equation*}
    \bar{\mu_2} = \SI{0.564+-0.044}{\per\centi\metre}. 
\end{equation*}

Der Würfel wird mit einer Abweichung von $\SI{0.9}{\percent}$ Eisen zugeordnet.\\

Für den Absorptionskoeffizient des zweiten homogenen Würfeld ergibt sich unter Verwendung der Werte der Nullmessung 

\vspace{-5pt}
\begin{equation*}
    \bar{\mu_3} = \SI{0.105+-0.006}{\per\centi\metre}. 
\end{equation*}

Der Würfel wird mit einer Abweichung von $\SI{9.5}{\percent}$ CH20 zugeordnet.\\

Im letzten Versuchsteil wird ein Würfel untersucht, der aus verschiedenen Materialien zusammengesetzt ist. Die Zuordnung 
dieser Materialien findet sich in Tabelle \ref{tab:mat}. Es kann allerdings keine Aussage darüber getroffen werden, ob diese
Zuordnungen tatsächlich stimmen. Teilweise gibt es große Abweichungen, die durch verschiedene Faktoren erklärt werden können. 
Zum einen hat der Strahl eine endliche Ausdehnung, wodurch bei einem diagonalen Durchgang nicht verhindert werden kann, dass 
der Strahl auch falsche Elementarwürfel durchläuft. Außerdem kann es zu Ungenauigkeiten der Justierung kommen, oder es können
sich Messungenauigkeiten in der Messung der Zeitintervalle finden. 

