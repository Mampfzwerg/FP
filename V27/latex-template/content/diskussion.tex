\section{Diskussion}
\label{sec:Diskussion}

Der experimentell ermittelte Wert für den Lande-Faktor der roten Cadmium-Linie liegt bei 

\begin{equation*}
    g_\text{rot} = \num{1.04+-0.18}.
\end{equation*}

Der Theoriewert für linearpolarsiertes Licht der roten Cadmium Linie liegt bei $g_\text{theo, rot} = 1$. Damit ergibt 
sich eine Abweichung von $\SI{4}{\percent}$.  

Der experimentell ermittelte Wert für den Lande-Faktor für zirkular polarisiertes Licht der blauen Cadmium-Linie liegt bei 

\begin{equation*}
    g_\text{blau, \sigma} = \num{1.85+-0.18}.
\end{equation*}

Der Theoriewert für zirkularpolarsiertes Licht der blauen Cadmium Linie liegt bei $g_\text{theo, blau, \sigma} = 1.75$. Damit ergibt 
sich eine Abweichung von $\SI{5.7}{\percent}$.

Der experimentell ermittelte Wert für den Lande-Faktor für linear polarisiertes Licht der blauen Cadmium-Linie liegt bei 

\begin{equation*}
    g_\text{blau, \pi} = \num{0.91+-0.08}
\end{equation*}

Der Theoriewert für linearpolarsiertes Licht der blauen Cadmium Linie liegt bei $g_\text{theo, blau, \sigma} = 0.5$. Damit ergibt 
sich eine Abweichung von $\SI{82}{\percent}$.

Die beiden ersten Versuchsteile haben dabei sehr gut funktioniert, da die Abweichungen sehr klein sind. Bei der Bestimmung des 
Lande-Faktors des linearpolarsierten Lichtes der blauen Cadmium-Linie trat eine deutlich größere Abweichung auf. Dafür kommt als 
Fehlerquelle das Magnetfeld in Betracht. Dieses konnte insbesondere bei der linearpolarsierten blauen Linie nicht 
hoch genug eingestellt werden. Somit ist keine optimale Trennung der Linien vorhanden. Außerdem sind weitere Fehlerquellen in
unzureichender Justierung zu verorten. 

